
\subsection{Quantum Bayesian Networks}

\label{subsec:int_QBN}

The use of graphs and visual depictions devised to work with quantum mechanics allows for an abstraction from the mathematical formulas behind the systems. This is the idea behind the Feynman Diagrams, that have been used extensively as David Kaiser confirms\cite{Kaiser2005}. Using a \ac{QBN} could provide an interesting way to represent data, namely conditional dependencies, and therefore be a valuable tool when dealing with quantum probabilities in large systems.  

A proposition for a \ac{QBN} model was first mentioned by Tucci\cite[1997]{Tucci1997}. The motivation behind \acs{QBN} would be the construction of a framework to calculate quantum mechanical conditional probabilities, and the followed approach was to alter the minimum possible its classic counterpart so that it would allow for working with quantum mechanics.

The main idea featured in this first approach\cite{Tucci1997} dealt with quantum pure states. According Tucci, ``a QB net for a pure state consists of a directed acyclic graph (DAG) and a transition matrix (a complex matrix), assigned to each node of the graph" \cite{Tucci2012} and that ``keeping QB nets[Quantum Bayesian Networks] close to CB nets[Bayesian Networks] can be very fruitful, because much is already known about CB nets"\cite{Tucci2012}. It's also enforced that each node has a numerical value and the whole graph also possesses a numerical value (the product of nodes), so factorization should be similar to their classical version.
The \ac{QBN} would be \ac{DAG} labelled with a collection of node matrices. Each node has a random variable attached to it and a matrix containing probability amplitudes (complex numbers). 
In its introductory article regarding Quantum Bayesian Networks, Tucci writes that ``keeping QB nets close to CB nets can be very fruitful, because much is already known about CB nets"\cite{Tucci2012}. Tucci also points an example of usage of \ac{QBN} in medical diagnosis\cite{Tucci2008}.


Meanwhile other approaches were formulated namely Leifer's\cite{Leifer2008} in his article "Quantum Graphical Models for Belief Propagation", that focus on attributing a density matrix to each node. This would be the equivalent of attributing a known probability function (like a Gaussian distribution), to each variable. 



% Examples!!!!