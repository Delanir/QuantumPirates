\section{Motivation}
\label{sec:int_motivation}


\begin{comment}
This study was motivated by attempts to devise a computational model for humans' inferential reasoning, namely, the mechanism by which people integrate data from multiplesources and generate a coherent interpretation of that data. J. Pearl, “Bayesian Networks: a model of self-activated: memory for evidential reasoning,” 1985.

% a state of mind in which there is knowledge of one's own existence and of the existence of surroundings antónio damásio (self comes to mind)

Under this assumptions, we, as humans, use a superposition of states in our inference process. This could be translated by saying that we live in an ambiguity of states and only when the self (our conscience), analyses we measure and decide which is in fact our belief.


\end{comment}


The history of Artificial Inteligence is strongly intertwined with the search for the principles that humans use to reason, as its main motivation is to try to transpose the same mechanisms to improve the computation in artificial systems. Numerous approaches to this issue have been devised, all with its merits and flaws, from the symbolic systems (that assumed that inteligence could be reduced to symbol manipulation), to a connectionist approach (where inteligence is deemed as an emergent property of simple units interacting with eachother). But some of the high hopes and expectations hold in the begining of this area, have been, to some extent not met. 
With the search for new paradigms in computing, for example the quantum computation, the idea of using those paradigms to better model the human's inference process, and by doing so to create new perpectives that can enrich artificial inteligence, seems to be only a natural approach.
 
As Pearl\cite{Pearl1985} writes, the original motivation behind the Bayesian Network model was " (...) to devise a computational model for human's inferential reasoning, namely, the mechanism by which people integrate data from multiplesources and generate a coherent interpretation of that data.". With this study we look forward to explore if the use of quantum mechanics principles in bayesian networks can hold better results than the classical approach. So, in a sense, this statement also ends up being the main motivation in this study. 
 
There is empirical evidence that our beliefs could be more accurately explained by a quantum model than a classical model, as there are numerous paradoxes that arise when we apply classical probability frameworks to describe human inference process\cite[Busemeyer et al]{Busemeyer2009}. 
Paradoxes such as the "Allais paradox" and the "Ellsberg paradox", point for inconsistencies between the expected utility hypotesis and experimental results\cite{Aerts2011}. 

The expected utility hypotesis has a strong mathematical foundation in the utility theory and yet "It is indeed well known that concepts combine in human minds in such a way that they show deviations from the expectations thatcould be drawn in classical set and probability theories. Analogously, subjects take decisions which seem to
contradict classical logic and probability theory."\cite{Aerts2011}. 
For example, the commutative property of events relating to the disjunction, does not accomodate the fact that it is verified\cite{TruebloodJ} that in human inference process, order affects the probability of a hypothesis given a sequence.


This makes the Quantum Cognition field arise as an area that applies the formalism of quantum theory to model to interpret cognitive phenomena. 

Some of the reasons pointed by Trueblood and Busemeyer\cite{Trueblood} to adopt a quantum based aproach for cognitive science would culminate in an approach to the inference process as a construction where ambiguity can be felt, and interference between beliefs and uncertainty are possible. 

The idea of resorting to quantum mechanics to explain phenomena underlying consciousness has been pointed in literature by the founding fathers of quantum mechanics, and starts to aquire importance in the neuroscience community, that seeks and recognizes that the analysis of dendrite-synapse fall into the spatio-temporal level explained by quantum mechanics\cite{Tarlaci2010}.

In this context we shall present a comparative study regarding quantum probability and bayesian probability applied to conditional reasoning in Bayesian Networks. 

There is already some work conducted, for example Pothos and Busemeyer \cite{Pothos2009} did a comparative study between a quantum cognition model and an equivalent Markov model to model and explain the violations of the "sure thing principle" in rational decision theory. 
One of the main conclusion of the study is the fact that the quantum cognition approach provides empirical findings that pose a challenge to a classical model.

Tucci\cite{Tucci2008} also devises an approach using Quantum Computers and Quick Medical Reference to give diagnosis. The Quick Medical Reference can be viewed as a large bayesian network, so the solution proposed comprises the modelation of the QMR as a \ac{QBN}.

