\begin{abstract}

In this document, we develop a model and a simulation of a quantization scheme for the mathematical puzzle created by Omohundro and Stewart - ``A puzzle for pirates '', also known as Pirate Game. This game is a multi-player version of the game ``Ultimatum '', where the players (Pirates), must distribute fixed number of gold coins acording to some rules.

The Quantum Theory of Games is a field that seeks to introduce the mathematical formalism of Quantum Mechanics in order to explore models of conflict that arise when rational beings make decisions. These models of conflict are pervasive in the structural make-up of our society. The combination of game theory and Quantum Probability, despite not having a practical application, can help in the development of new quantum algorithms. Furthermore the fact that Game Theory is transversal to many areas of knowledge can provide insights to future application of these models.

In this dissertation we focused on the role of quantum entanglement and the use of quantum strategies in the game system. We found that when there is no entanglement the game behaved as the original problem even when the players adopted quantum strategies. When using a unrestricted strategic space and the game system is maximally entangled we found that the game is strictly determined (like the original problem). We also found that when only a the captain has access to quantum strategies in the Pirate Game, she can obtain all the gold coins. These results corroborate similar findings in the field.


\end{abstract}