\begin{resumo}

Neste trabalho desenvolvemos e simul\'{a}mos um modelo qu\^{a}tico para o puzzle matem\'{a}tico criado por Omohundro e Stewart, ``Um puzzle para piratas''(original em ingl\^{e}s ``A Puzzle for Pirates''). Este jogo consiste numa vers\~{a}o multi-jogador do jogo ``Ultimato'', no qual os jogadores (Piratas), distribuem um número limitado de moeadas de ouro.

A Teoria de Jogos Qu\^{a}ntica \'{e} uma \'{a}rea que procura introduzir o formalismo matem\'{a}tico na base da Mec\^{a}nica Qu\^{a}ntica para explorar modelos de conflito que surgem quando seres racionais tomam decis\~{o}es. Estes modelos de conflito est\~{a}o na base da estrutura da nossa sociedade. A combina\c{c}\~{a}o de Teoria de Jogos e a Teoria de Probabilidade Qu\^{a}ntica apesar de ainda n\~{a}o ter uma aplica\c{c}\~{a}o pr\'{a}tica pode ajudar no desenvolvimento de novos algoritmos qu\^{a}nticos. 

Nesta disserta\c{c}\~{a}o foc\'{a}mo-nos sobretudo no papel do fen\'{o}meno qu\^{a}ntico entrela\c{c}amento e exist\^{e}ncia de estrat\'{e}gias qu\^{a}nticas no sistema do jogo. Verific\'{a}mos que este fen\'{o}meno introduz varia\c{c}\~{o}es na utilidade esperada pelos jogadores, para algumas estrat\'{e}gias \`{a} semelhan\c{c}a de outros modelos na \'{a}rea. Quando n\~{a}o existe entrela\c{c}amento as estrat\'{e}gias qu\^{a}nticas portam-se como as estrat\'{e}gias cl\'{a}ssicas do jogo original. 



\end{resumo}