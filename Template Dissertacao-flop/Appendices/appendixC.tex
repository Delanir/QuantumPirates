\fancychapter{Quantum Roullete}
\label{ap:c}


\begin{lstlisting}
function []= quantum_roullete3()
%%
% Simulation based on
% S. Salimi and M. M. Soltanzadeh, "Investigation of quantum roulette arXiv : 0807 . 3142v3 [ quant-ph ] 30 Apr 2009," 2009.

%%

N=3
%% make states
I = eye(N);
D = (1/N)*ones(N);


%% permutation matrices change between states
 X0 = circshift(I, 0);
%X0 = I;
 X1 = circshift(I, 1);
%X1 = [0 1 0; 1 0 0; 0 0 1];
 X2 = circshift(I, 2);
%X2 = [0 0 1; 0 1 0; 1 0 0];
 X3 = circshift(I, 3);
%X3 = [1 0 0; 0 0 1; 0 1 0];
X4 = circshift(I, 4);
%X4 = [0 0 1 ; 1 0 0; 0 1 0];
 X5 = circshift(I, 5);
%X5 = [0 1 0; 0 0 1; 1 0 0];


%%Fourier Matrix
F= (1/sqrt(N))*fft(I)

T0 = circshift(F,[0 0]);
T1 = circshift(F,[0 1]);
T2 = circshift(F,[0 2]);
%%

%%Step1

%Assuming Alice places the roullete in state 2 were both player can see
Ro0= [0; 1; 0]*[0; 1; 0]'

%%Step2

%As Alice chose state 2, Bob will select T1 to rotate the state
Ro1 = T1 * Ro0 * T1'

%step 3 -
Ro1 = T2 * Ro1 * T2'

%%Step3
%Alice will play again 
%

Ro2 = (2/6)*X0*Ro1*X0' + (1/12)*X1*Ro1*X1' +(1/12)*X2*Ro1*X2' +(1/6)*X3*Ro1*X3' +(1/6)*X4*Ro1*X4' +(1/6)*X5*Ro1*X5'

%%Step4
%Bob can choose which state he wants

Ro3 = T0'*Ro2*T0

Ro3 = T1'*Ro2*T1

Ro3 = T2'*Ro2*T2

Ro4 = (2/6)*X0*Ro3*X0' + (1/12)*X1*Ro3*X1' +(1/12)*X2*Ro3*X2' +(1/6)*X3*Ro3*X3' +(1/6)*X4*Ro3*X4' +(1/6)*X5*Ro3*X5'

Ro3 = T0'*Ro3*T0
end
\end{lstlisting}