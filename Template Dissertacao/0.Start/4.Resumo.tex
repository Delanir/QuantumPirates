\begin{resumo}

Neste trabalho desenvolvemos e simul\'{a}mos um modelo qu\^{a}tico para o puzzle matem\'{a}tico criado por Omohundro e Stewart, ``Um puzzle para piratas''(original em ingl\^{e}s ``A Puzzle for Pirates''). Este jogo consiste numa vers\~{a}o multi-jogador do jogo ``Ultimato'', no qual os jogadores (Piratas), distribuem um n\'{u}mero limitado de moeadas de ouro.

A Teoria de Jogos Qu\^{a}ntica \'{e} uma \'{a}rea que procura introduzir o formalismo matem\'{a}tico na base da Mec\^{a}nica Qu\^{a}ntica para explorar modelos de conflito que surgem quando seres racionais tomam decis\~{o}es. Estes modelos de conflito est\~{a}o na base da estrutura da nossa sociedade. A combina\c{c}\~{a}o de Teoria de Jogos e a Teoria de Probabilidade Qu\^{a}ntica apesar de ainda n\~{a}o ter uma aplica\c{c}\~{a}o pr\'{a}tica pode ajudar no desenvolvimento de novos algoritmos qu\^{a}nticos. 

Nesta disserta\c{c}\~{a}o foc\'{a}mo-nos sobretudo no papel do fen\'{o}meno qu\^{a}ntico entrela\c{c}amento e exist\^{e}ncia de estrat\'{e}gias qu\^{a}nticas no sistema do jogo. Verific\'{a}mos que quando n\~{a}o existe entrela\c{c}amento o jogo se comporta como um jogo cl\'{a}ssico, mesmo quando os jogadores utilizam estrat\'{e}gias qu\^{a}nticas. Quando utilizamos um espa\c{c}o estrat\'{e}gico n\~{a}o restrito e o sistema est\'{a} maximamente entrela\c{c}ado descobrimos que o jogo \'{e} estrictamente determinado (como no problema original). Tamb\'{e}m se verificou que quando apenas o capit\~{a}o tem acesso a estrat\'{e}gias qu\^{a}nticas no jogo, este consegue obter todas as moedas. Estes resultados corroboram resultados similares na literatura.



\end{resumo}