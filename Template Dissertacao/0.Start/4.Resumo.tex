\begin{resumo}

Neste trabalho desenvolvemos e simul\'{a}mos um modelo qu\^{a}tico para o puzzle matem\'{a}tico criado por Omohundro e Stewart, ``Um puzzle para piratas''(original em ingl\^{e}s ``A Puzzle for Pirates''). Este jogo consiste numa vers\~{a}o multi-jogador do jogo ``Ultimato''. 

A Teoria de Jogos Qu\^{a}ntica \'{e} uma \'{a}rea que procura introduzir o formalismo matem\'{a}tico na base da Mec\^{a}nica Qu\^{a}ntica para explorar modelos de conflito que surgem quando seres racionais tomam decis\~{o}es. Estes modelos de conflito est\~{a}o na base da estrutura da nossa sociedade. A combina\c{c}\~{a}o de Teoria de Jogos e a Teoria de Probabilidade Qu\^{a}ntica apesar de ainda n\~{a}o ter uma aplica\c{c}\~{a}o pr\'{a}tica pode ajudar no desenvolvimento de novos algoritmos qu\^{a}nticos. O facto da Teoria de Jogos ser ruma disciplina transversal a muitas \'{a}reas do conhecimento pode fazer com que estes modelos possam eventualmente vir a ter relev\^{a}ncia.

Foc\'{a}mo-nos sobretudo no papel do fen\'{o}meno qu\^{a}ntico entrela\c{c}amento no sistema do jogo. Verific\'{a}mos que este fen\'{o}meno introduz varia\c{c}\~{o}es na utilidade esperada pelos jogadores, para algumas estrat\'{e}gias \`{a} semelhan\c{c}a de outros modelos na \'{a}rea. Contudo também verific\'{a}mos a exist\^{e}ncia de estrat\'{e}gias nas quais n\~{a}o existe interfer\^{e}ncia.



\end{resumo}