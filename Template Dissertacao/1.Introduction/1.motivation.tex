\section{Motivation}
\label{sec:int_motivation}



It's not always easy to understand and introduce new paradigms. Some things we find natural nowadays, were once the source of controversy. Taking as an example the number zero $0$, as numbers were introduced to help count physical objects, the idea of representing ``nothingness'' was once considered strange. In the begining there were numerous ways devised t deal with this mathematical inconvinience, a speacial case. However the need to use ``zero'' as a number in its own right lead to the popularization of its concept as a number and opened door to new breakthroughs in mathematics\cite{Kaplan2000}. 

Quantum mechanics shared this same problem and the mathematical formalization grew out of the need to explain phenomena\cite{Mehra1982}. Nowadays we currentely accepted quantum mechanics as a tool, while still trying to understand why it works at an epistological level.

The idea of trying to do some contribution in quantum computing arised from the desire to explore this paradigm. The standard curriculum of a computer scientist undergraduate is focused on the current computational paradigm, which has its roots in von Neumman Architecture\cite{neumann45edvac}. While the construction and the innerworkings of computational systems that rely on new paradigms may fall outside of the scope of computer science, understanding and taking advantage from the point of view of Information Technology and pushing boundaries on model representation are areas where computer scientist might contribute.



