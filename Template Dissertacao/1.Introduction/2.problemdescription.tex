\section{Problem Description}
\label{sec:int_problem}

A model provides a way to abstract the reality, which is often too complex to analyse in its breadth. The Game Theory is an area that tried to find ways to represent and analyse situations of conflict generated when intelligent rational beings make decisions. This discipline has major applications in Economics, Biology, Political Science, and Artificial Intelligence. 
Combining both the mathematical foundations of Quantum Mechanics, and the Game Theory is an idea that is starting to attract attention. These two areas share the same founding father, von Neumman, and creating a Quantum Model for a game seems to be a relevant way to explore the theory behind Quantum Mechanics while having a controlled and creative way to apply it. It is also interesting to analyse how these quantum games differ from their well defined classical counterparts. Furthermore  simulating quantum algorithms on a classical computer is usually impractical because this systems grow exponentially with the number of qubits. In game theory we have relevant problems that can be modelled using few qubits, thus making it possible to simulate in a classical computer.

The Pirate Game is a mathematical, Game Theory, problem; with this work want to explore it in the light of quantum game theory. This means developing a quantization scheme, which is a way to transform the original game in a quantum simulation. Our simulations will be implemented on Matlab.

Despite not having a clear ``real-world'' application yet, modelling games with quantum mechanics rules may aid the development of new algorithms that would be ideally deployed using quantum computers. 

Furthermore applying quantum probabilities to a well established area as Game Theory, which has applications in fields such as Economy, Political Science, Psychology, Biology, might introduce new insights and even relevant practical applications\cite{Eisert2008}. 



