


\section{Quantum Computing}
\label{sec:int_quantum_computing}


%\subsection{Introduction}
%\label{subsec:int_int_quantum_computing}



The quantum equivalent of a bit (the basic unit of information in computers), is a qubit (or quantum bit). A qubit is a two-state quantum system that can be interpreted as normalized vectors in a $2$-dimensional Hilbert space. This Hilbert Space($\mathcal{H}^2$), an element in that space can be uniquely specified by resorting to two orthonormal basis (also known as pure states). In quantum computing, the basis used are ${ \vert 0 \rangle , \vert 1 \rangle }$ or $\left[\begin{array}{c}
1\\
0
\end{array}\right]$
 and 
$\left[\begin{array}{c}
0\\
1
\end{array}\right]$ .

\subsection{Superposition}
\label{subsec:superposition}

The qubit can be described in terms of linear transformations of pure states as in \ref{qubit}, $\omega_{0}$ and $\omega_{1}$ are complex numbers called probability amplitudes. When a system is in a mixture of this pure states (as the general representation in \ref{qubit} might point), it is in a phenomenon known as superposition. Measuring forces the system to collapse and assume one of the pure states with a certain probability. 

\begin{equation}
\label{qubit}
\vert \psi \rangle = \omega_{0}\vert0\rangle+\omega_{1}\vert1\rangle = \left[\begin{array}{c}
\omega_{0}\\
\omega_{1}
\end{array}\right]
\end{equation}

In the example stated in the representation \ref{qubit} the probability of the system falling into the state $\vert 0 \rangle $ would be $\vert\omega_{0}\vert^{2}$ . The probability of the system falling into $\vert 1 \rangle $ would be $\vert\omega_{1}\vert^{2}$ . The second axiom of Kolmogorov (law of total probability) is verified (\ref{2nd_axyom}).

\begin{equation}
\label{2nd_axyom}
\vert\omega_{0}\vert^{2}+\vert\omega_{1}\vert^{2}=1
\end{equation}

%So considering again the inner product  ($\langle \phi \vert \psi \rangle$) we can verify that it holds a physical interpretation: the probability amplitude of the state %$\psi$ collapsing into $\phi$ .
 
$\omega_{0}$ and $\omega_{1}$ (\ref{qubit}) are complex numbers, the so-called probability amplitudes. When squared, the probability amplitude represents a probability. 

\subsection{Operators}
\label{subsec:QCoperators}

In order to perform transformations in the qubit systems, we can define linear operators in the Hilbert space. One of the most important classes of operators being the self-adjoint operators,  $A = A^{*}$, that have the property stated in Equation \ref{eq_adjoint}. The Hermitian operator is one that satisfies the property of being equal to its conjugate transposed, $A = A^{*T} =A^\dagger$ .  In a finite-dimensional Hilbert space defined by a set of orthonormal basis every self-adjoint operator is Hermitian. 

\begin{equation}
\label{eq_adjoint}
\langle A^{*}z\vert x\rangle=\langle z\vert Ax\rangle
\end{equation}

Unitary and Hermitian operators are used in the majority of quantum algorithms because they can be reversible, and because they insure that no rule of quantum mechanics is violated while applying the transformations. For that matter we arrive at an interesting difference between classical computing and quantum computing: it is impossible to copy or clone unknown quantum states\cite{Rieffel2011}. This is known as: The No-Cloning Principle.

We can prove the ``The No-Cloning Principle'' by \textit{reductio ad impossibilem}. Suppose we have an operator $U$ which is unitary and clones quantum states, and two unknown quantum states $\vert a \rangle$ and $\vert  b \rangle$. The transformation $U$ means that if we apply it to $\vert a \rangle$ we have $U(\vert a \rangle\vert 0 \rangle)=\vert a \rangle\vert a \rangle$. The result when applying to $\vert b \rangle$ is  $U(\vert b \rangle\vert 0 \rangle)=\vert b \rangle\vert b \rangle$. 

If we consider a state  $\vert c \rangle = \frac{1}{\sqrt{2}}( \vert a \rangle + \vert b \rangle)$, by the principle of linearity $U(\vert c \rangle\vert 0 \rangle)=\frac{1}{\sqrt{2}}( U(\vert a \rangle\vert 0 \rangle) +U(\vert b \rangle\vert 0 \rangle))$, giving the final result $U(\vert c \rangle\vert 0 \rangle)=\frac{1}{\sqrt{2}}( \vert a \rangle\vert a \rangle + \vert b \rangle\vert b \rangle)$. However if $U$ is a cloning operator then $U(\vert c \rangle\vert 0 \rangle)=\vert c \rangle\vert c \rangle$. 

$\vert c \rangle\vert c \rangle = \frac{1}{\sqrt{2}}( \vert a \rangle\vert a \rangle + \vert a \rangle\vert b \rangle+ \vert b \rangle\vert a \rangle+ \vert b \rangle\vert b \rangle)$ is different from $\frac{1}{\sqrt{2}}( \vert a \rangle\vert a \rangle + \vert b \rangle\vert b \rangle)$, thus we can affirm there is no unitary operator $U$ that can clone unknown quantum states\cite{Rieffel2011}.

When presented with $q$ qubits, a $q$-dimensional square matrix is called a Quantum Gate.



Some important Quantum Gates that operate on a single qubit are:

\begin{itemize}
\item The Identity Matrix \ref{eq:idims}.
\begin{equation}
\label{eq:idims}
I=\left[\begin{array}{cc}
1 & 0\\
0 & 1
\end{array}\right]
\end{equation}

\item The Pauli Operators. This operators are used to perform the NOT operation. 
	\begin{itemize}
		\item The Bit-Flip Operator\begin{equation}
\label{eq:idims_1}
X=\left[\begin{array}{cc}
0 & 1\\
1 & 0
\end{array}\right]
\end{equation}
		\item The Phase-Flip Operator\begin{equation}
\label{eq:idims_2}
Z=\left[\begin{array}{cc}
1 & 0\\
0 & -1
\end{array}\right]
\end{equation}
\item The Bit and Phase-Flip Operator\begin{equation}
\label{eq:idims_3}
Y=\left[\begin{array}{cc}
0 & -i\\
i & 0
\end{array}\right]
\end{equation}
	\end{itemize}
\item The Hadamard Gate, belongs to a general class of Fourier Transforms. This $2\times 2$ particular case is also a Discrete Fourier Transform matrix.
\begin{equation}
\label{eq:idimssd_3}
H=\frac{1}{\sqrt{2}}\left[\begin{array}{cc}
1 & 1\\
1 & -1
\end{array}\right]
\end{equation}
\end{itemize}


\begin{comment}
Observables are deemed as the physical properties that can be measured. One way to think of them is to consider the 20 question game, where a two player game where one person thinks of an object and then the second person has a set of 20 "Yes" or "No" to discover the object, one observable could be "Is it red?". With each question the space comprising the possible answers will not increase, this means that asking two times in a row if the object is red in the game will always yield the same answer.
 
Hermitian operators are suitable for being used as observables, as their eigenvalues are real numbers.
The eigenvectors associated with the eigenvalues of the observable will correspond to the state in which the system will be after applying the Hermitian operator. Thus applying an observable to the system can be viewed as doing a projection of the system in the basis formed by the eigenvectors.  
\end{comment}

\subsection{Compound Systems}
\label{subsec:compound_systems}

The representation of a system comprising multiple qubits grows exponentially. If to represent a single qubit system there is a $2$-dimensional Hilbert space($\mathcal{H}^{2}$), to represent a system with m qubits a $2^m$-dimension space would be required. 
To represent a higher dimension multiple-qubit system composed by single-qubits, one can perform a tensor product of single-qubit systems.

%Tensor product
The tensor product is an operation denoted by the symbol $\otimes$. 
Given two vector spaces V and W with basis 
\ref{eq:qc_cs_vspace} and \ref{eq:qc_cs_wspace}
respectively, their tensor product would be the $mn$-dimensional vector space with a basis with elements from the set \ref{eq:qc_cs_vwspace} \cite{Rieffel2011}.


\begin{equation}
\label{eq:qc_cs_vspace}
A = \left\{ \vert \alpha_{1} \rangle, \vert \alpha_{2} \rangle , ..., \vert \alpha_{m} \rangle \right\}
\end{equation} 

\begin{equation} 
\label{eq:qc_cs_wspace}
B = \left\{ \vert \beta_{1} \rangle, \vert \beta_{2} \rangle , ..., \vert \beta_{n} \rangle \right\}
\end{equation} 

\begin{equation} 
\label{eq:qc_cs_vwspace} 
C =\left\{  \vert \alpha_{i} \rangle \otimes \vert \beta_{j} \rangle \right\}
\end{equation}

For example, if we consider two Hilbert spaces $\mathcal{H}^2$ with basis
$ A=\left\{ \vert 0 \rangle , \vert 1 \rangle \right\}$ and 
$B =\left\{ \vert - \rangle, \vert + \rangle \right\}$
, their tensor product would be a $\mathcal{H}^4$ with basis \ref{eq:qc_cs_vwsupahspace}.
\begin{equation}
\label{eq:qc_cs_vwsupahspace}
AB = \left\{ \vert 0  - \rangle, \vert 0 + \rangle, \vert 1 - \rangle,  \vert 1 + \rangle \right\}
\end{equation}

Now taking the former Hilbert space, supposing we have the qubits \ref{eq:qc_cs_vqubit} and \ref{eq:qc_cs_wqubit}.

\begin{equation}
\label{eq:qc_cs_vqubit}
\vert v \rangle = a_{0}\vert 0\rangle + a_{1}\vert 1 \rangle
\end{equation}

\begin{equation}
\label{eq:qc_cs_wqubit}
\vert w \rangle = b_{0}\vert -\rangle + b_{1}\vert + \rangle
\end{equation}

\begin{equation}
\label{eq:qc_cs_vwsupahspace2}
\vert v \rangle \otimes \vert w \rangle = a_{0}b_{0}\vert 0  - \rangle + a_{0}b_{1}\vert 0 + \rangle + a_{1}b_{0}\vert 1 - \rangle +  a_{1}b_{1}\vert 1 + \rangle 
\end{equation}

The Bra-ket notation provides a way to prevent the escalation of the basis notation. When specified the vector space the basis can be specified in base 10 for simplicity sake. According to this the basis of the last example would be $AB = \left\{ \vert 0 \rangle, \vert 1 \rangle, \vert 2 \rangle,  \vert 3 \rangle \right\}$ and a system with $3$ qubit could be represented in a Hilbert space, $\mathcal{H}^{8}$, with basis $H= \left\{ \vert 0 \rangle, \vert 1 \rangle, \vert 2 \rangle,  \vert 3 \rangle,  \vert 4 \rangle,  \vert 5 \rangle,  \vert 6 \rangle,  \vert 7 \rangle \right\}$.



\subsection{Entanglement}
\label{subsec:entanglement}

In multiple qubit systems, qubits can interfere with each other, thus making impossible to determine the state of part of the system without ``disturbing" the whole. In other words, there are states in a multi-qubit system that cannot be described as a probabilistic mixture of the tensor product of single-qubit systems; when this happens this state is not separable (with respect to the tensor product decomposition), this phenomenon is called quantum entanglement\cite{Rieffel2011}. If a mixed quantum state $\psi$ in a quantum system constructed by $V_{1}, V_{2}, ..., V_{n}$ is separable it can be written as:


\begin{equation}
\vert\psi\rangle= \sum^m_{j=1}{p_{j}\vert\varphi_{j}^{1}\rangle\langle\varphi_{j}^{1}\vert \otimes ... \otimes \vert\varphi_{j}^{n}\rangle\langle\varphi_{j}^{n}\vert}, \sum_{i}{p_{i}}=1, \vert \varphi_{j}^{i} \rangle \in V_{i}
\end{equation}

Quantum entanglement is one of the main differences from the classical theory\cite{Rieffel2011}. In an entangled pair each member is described with relation to the other members. This property is not local as transformations that act separately in different parts of an entangled system cannot break the entanglement. However if we measure a part of an entangled system  the system collapses, and if we measure the other part in any point of time from that moment we will find a correlation with the outcome of the first measurement.

For example, supposing we consider the following quantum states, known as Bell states:
\begin{equation}
\vert\Phi^{+}\rangle=\frac{1}{\sqrt{2}}(\vert0\rangle_{A} \otimes\vert0\rangle_{B} +\vert1\rangle_{A} \otimes\vert1\rangle_{B})=\frac{1}{\sqrt{2}}(\vert00\rangle+\vert11\rangle)
\end{equation}

\begin{equation}
\vert\Phi^{-}\rangle=\frac{1}{\sqrt{2}}(\vert0\rangle_{A} \otimes\vert0\rangle_{B} -\vert1\rangle_{A} \otimes\vert1\rangle_{B})=\frac{1}{\sqrt{2}}(\vert00\rangle-\vert11\rangle)
\end{equation}

\begin{equation}
\vert\Psi^{+}\rangle=\frac{1}{\sqrt{2}}(\vert0\rangle_{A} \otimes\vert1\rangle_{B} +\vert1\rangle_{A}\otimes\vert0\rangle_{B})=\frac{1}{\sqrt{2}}(\vert01\rangle+\vert10\rangle)
\end{equation}

\begin{equation}
\vert\Psi^{-}\rangle=\frac{1}{\sqrt{2}}(\vert0\rangle_{A} \otimes\vert1\rangle_{B} -\vert1\rangle_{A} \otimes\vert0\rangle_{B})=\frac{1}{\sqrt{2}}(\vert01\rangle-\vert10\rangle)
\end{equation}

These states form a particular basis in $\mathcal{H}^4$ as they are all entangled states and they are maximally entangled. If we have a system of two qubits in a Hilbert space in the mixed state $\vert\Phi^{+}\rangle$ and we measure the qubit A (by deciding the outcome $0$ or $1$ with a probability of $0.5$ for each),we are automatically uncovering the value of the qubit B. In this state we know the second qubit will always yield the same value of the first measured qubit. The second value is correlated with the first one.




