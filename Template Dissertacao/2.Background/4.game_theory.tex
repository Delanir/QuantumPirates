\section{Game Theory}
\label{sec:background:game_theory}

Game Theory did not exist in its own right before von Neumann published the article " On the Theory of Games of Strategy''. This field deals mainly with the study of interactions between racional decision-makers\cite{Neumann1944}. Game Theory knows numerous applications in areas such as Economics, Political Science, Biology, and Artificial Inteligence. 

\subsection{Definition of a Game}
\label{subsec:background:game_theory_definition}

The concept of utility, expected utility, or payoff derives from the theory that we can use real numbers to represent the wants and needs of the players. This allows to compare different states and rank preferred outcomes\cite{Neumann1944}.

There are two standard representations of games:
\begin{itemize}
\item Normal Form - lists what payoffs the players get as a function of their actions as if they all make their moves simultaneously. 
\item Extensive Form - extensive form games can be represented by a tree and represent timed actions.
\end{itemize}

A finite game is a game that has a finite set of actions, a finite number or players, and it does not go on indefinitely.

A zero-sum game is a mathematical representation of a system where the gains are completely evened out by the losses; this means that the sum of the utilities of all players will always be zero.



\subsection{Nash Equilibrium}
\label{subsec:background:game_theory_nash_equilibrium}

When all players cannot improve their utility by changing their strategy unilateraly, we have an equilibrium point. This equilibrium point is named after John Nash, who proved that it exists at least one mixed strategy Nash equiibrium in a finite game\cite{nash50}\cite{Nash51}. This concept is used to analise game where several decision makers interact simultaneously and the final outcome depends on the players strategy.

In a sequential game 

\subsection{Pareto Optimal}
\label{subsec:background:game_theory_pareto_optimal}

When every player cannot improve her payoff without lowering another player's expected utility we have a Pareto Optimal solution.





\section{Quantum Game Theory}
\label{sec:background_quantum_game_theory}

[Intro SAUCEs]

The definition of to normal representation of extensive games proposes a representation
for finite extensive form games\cite{Fra2011}\cite{Fra2011a}. The representation assumes that there
are only two available actions, which can be represented by a qubit.
Those actions could be a yes/no decision or a cooperate/defeat as
those found in many classical game theory problems. A game in this
form is represented by a six-tuple\ref{eq:quantum_game_six_tuple},
where:

\begin{equation}
\Gamma=(\mathcal{H}^{a},\: N,\:\vert\psi_{in}\rangle,\:\xi,\:\{\mathcal{U}_{j}\},\:\{E_{i}\})\label{eq:quantum_game_six_tuple}
\end{equation}

\begin{itemize}
\item $a$ is the number of actions (qubits), in the game; 
\item $\mathcal{H}^{a}$ is a $a$-dimensional Hilbert space constructed
as $\otimes_{j=1}^{a}\mathbb{C}^{2}$, with basis $\mathcal{B}$;
\item $\vert\psi_{in}\rangle$ is the initial state of the compound-system
composed by $a$ qubits: $\vert\varphi_{1}\rangle,\:\vert\varphi_{2}\rangle, ..., \vert\varphi_{j}\rangle, ..., \vert\varphi_{a}\rangle$;
\item $\xi$ is a mapping function that assigns each action to a player;
\item For each qubit $j$ $\mathcal{U}_{j}$ is a subset of unitary operators
that can be used by the player to manipulate her qubit;
\item Finally, for each player $i$ $E_{i}$ is a utility functional that
specifies her payoff. This is done by attributing a real numbered utility to the measurent of the projection of a basis in the final state\ref{eq:quantum_game_definition_payoff_func}.\end{itemize}

\begin{equation}
E_{i}=\sum_{b \in \mathcal{B}} u_{i}(b)\vert \langle b\vert \psi_{fin}\rangle\vert^{2}, u_{i}(b) \in \mathbb{R}
\label{eq:quantum_game_definition_payoff_func}
\end{equation}

The strategy of a player $i$ is a map $\tau_{i}$ which assigns a
unitary operator $U_{j}$ to every qubit $j$ that is manipulated
by the player (\inputencoding{latin1}{$j$$\in\xi^{-1}(i)$}\inputencoding{latin9}).
The simultaneous move is represented in \ref{eq:quantum_game_definition_move}.

\begin{equation}
\vert\psi_{fin}\rangle=\otimes_{i=1}^{N}\otimes_{j\in\xi^{-1}(i)}\mathcal{U}_{j}\vert\psi_{in}\rangle\label{eq:quantum_game_definition_move}
\end{equation}

\section{Overview}
\label{sec:background_overview}

