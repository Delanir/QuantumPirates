\clearpage
\section{Game Theory}
\label{sec:background:game_theory}

Game Theory did not exist as a field of own right before von Neumann published the article " On the Theory of Games of Strategy''. This field deals mainly with the study of interactions between rational decision-makers\cite{Neumann1944}. Game Theory knows numerous applications in areas such as Economics, Political Science, Biology, and Artificial Intelligence. 

\subsection{Definition of a Game}
\label{subsec:background:game_theory_definition}

A game ($\Gamma$), is a model of conflict between players characterized by\cite{Osborne2004}\cite{OsbRub94}\cite{Fra2011}:
\begin{itemize}
\item A set of Players: $P=\{1, 2, .., N \}$.
\item For each player there is set of Actions.
\item There are preferences , for each player, over the set of Actions.
\end{itemize}


The preferences in the previous definitions are defined with resort to concept of utility, expected utility, or payoff derives from the theory that we can use real numbers (Equation \ref{gt:utilidadesreais}), to model and represent the wants and needs of the players. 

\begin{equation}
u:X\rightarrow\mathbb{R}
\label{gt:utilidadesreais}
\end{equation}

This simplified mathematical model allows to compare different states and rank preferred outcomes. If $u_{i}(x)$ is a utility function for player $i$, and $u_{i}(A)>u_{i}(B)$, that would mean that the player would strictly prefer $A$ over $B$. The concept of expected utility is fundamental to analyse games, as a rational player would try to maximize her expected utility\cite{Neumann1944}\cite{Osborne2004}\cite{Leyton-Brown2008:Essentials_Game_Theory}.


The options a player $i$ has when choosing an action, when the outcome depends not only on their choice (but also on the actions taken by the other players), is referred as strategy $s_{i}$. The set of strategies available to a player is represented by the set $S_{i}$.

When a strategy gives strictly a higher expected utility in comparison to other strategies, we have a strategy that dominates, or a dominant strategy. The mathematical definition of dominance is represented in Equation \ref{gt:dominant_strategy}; where, for a player $i$, in spite of the strategies chosen by the other players (denoted by $s_{-i}$), there is a strategy $s^{*}$ which gives always an higher expected utility in comparison with other strategies $s^{'}$ available to player $i$.

\begin{equation}
\forall s_{-i}\in S_{-i}[u_{i}(s^{*},s_{-i})>u_{i}(s^{'},s_{-i})]
\label{gt:dominant_strategy}
\end{equation}

The strategy that leads to the most favourable outcome for a player, taking into account other players strategies, is known as best response.

A pure strategy defines deterministically how the player will play the game. In the game represented in Table\ref{tab:2prisionersdillema_tab2} , in a pure strategy, the players either they choose ``Cooperate'' ($C$), or ``Defect'' ($D$). 

If there is a probability distribution associated with probability of playing with a determined pure strategy, we have a mixed strategy. 

There are two standard representations of games:
\begin{itemize}
\item Normal Form - lists what payoffs the players get as a function of their actions as if they all make their moves simultaneously. This games are usually represented by a matrix, for example \ref{tab:2prisionersdillema_tab2}.



\item Extensive Form - extensive form games can be represented by a tree and represent timed actions.
\end{itemize}

\begin{center}
\begin{table}[h]
\begin{centering}
\begin{tabular}{ccc}
\hline 
 & Player 2: C & Player 2: D\tabularnewline
\hline 
Player 1: C & (2,2) & (0,3)\tabularnewline
Player 1: D & (3,0) & (1,1)\tabularnewline
\hline 
\end{tabular}
\par\end{centering}

\caption{Example of a Normal Form game. This is a possible representation of the Prisoner's Dilemma game.}
\label{tab:2prisionersdillema_tab2}
\end{table}
\par\end{center}

A finite game is a game that has a finite set of actions, a finite number or players, and it does not go on indefinitely.

A zero-sum game is a mathematical representation of a system where the gains are completely evened out by the losses; this means that the sum of the utilities of all players will always be zero.





\subsection{Nash Equilibrium}
\label{subsec:background:game_theory_nash_equilibrium}

If we observe the representation of the classic game ``Prisoner's Dilemma'', we can observe that each player has two strategies; either they choose ``Cooperate'' ($C$), or ``Defect'' ($D$). The ``Defect'' strategy, for both players (the game is symmetrical), always yield a higher payoff in spite of the other player's strategy, this means that it is a dominant strategy and also constitutes a best response to the game. If both players chose their best response the final outcome $(D,D)$ becomes an equilibrium solution, more specifically a Nash Equilibrium. 

When all players cannot improve their utility by changing their strategy unilaterally, we have an equilibrium point. This equilibrium point is named after John Nash, who proved that it exists at least one mixed strategy Nash equilibrium in a finite game\cite{nash50}\cite{Nash51}. This concept is used to analyse game where several decision makers interact simultaneously and the final outcome depends on the players strategy\cite{Osborne2004}.

In order to compute a mixed strategy Nash Equilibrium in a 2-player 2 actions game, we need to define the probability distribution $P(C)=q$, $P(D)=1-q$ that makes player 1 indifferent to the whether player 2 decides to Cooperate or to Defect, and the probability distribution $P(C)=p$, $P(D)=1-p$ that makes player 2 indifferent to whether player 1 decides to Cooperate or to Defect. In the game represented in Table \ref{tab:2pennyflip_tab2} the  mixed strategies $(\frac{1}{3} , \frac{2}{3})$ and $(\frac{2}{3} , \frac{1}{3})$ are Nash Equilibria.

\begin{center}
\begin{table}[h]
\begin{centering}
\begin{tabular}{ccc}
\hline 
 & Player 2: C & Player 2: D\tabularnewline
\hline 
Player 1: C & (2,1) & (0,0)\tabularnewline
Player 1: D & (0,0) & (1,2)\tabularnewline
\hline 
\end{tabular}
\par\end{centering}

\caption{Example of a Normal Form game. The mixed strategies $(\frac{1}{3} , \frac{2}{3})$ and $(\frac{2}{3} , \frac{1}{3})$ are Nash Equilibria for this game. }
\label{tab:2pennyflip_tab2}
\end{table}
\par\end{center}

When there is at least a Nash Equilibrium when the players choose pure strategies we have a a strictly determined game\cite{Leyton-Brown2008:Essentials_Game_Theory}

In an extensive form game we have a concept of sub-game perfect Nash equilibrium, when a strategy is a Nash equilibrium for all sub-games in the original game\cite{Leyton-Brown2008:Essentials_Game_Theory}.

\subsection{Pareto Optimal}
\label{subsec:background:game_theory_pareto_optimal}

From the point of view of an observer outside the game system some outcomes may seem better than others. For example on the game represented in Table \ref{tab:2prisionersdillema_tab2} (Prisioners' Dilemma), the outcome $(C,C)$ seems better than the outcome $(D,D)$ because it provides a strictly higher utility to the players. However we know that the outcome $(D,D)$ is the Nash Equilibrium of the game. If both players use their best response their outcome might not be the best outcome for both players.

When every player cannot improve her payoff without lowering another player's expected utility we have a Pareto Optimal solution.

For example if we have two players and 10 units of a finite resource to distribute among the players, the Pareto optimal solution is $(5,5)$. Any other attempt to redistribute -$(6,4)$, $(1,9)$, etc...- would always leave a player worse than the Pareto optimal distribution of the resource.


\clearpage
\section{Quantum Game Theory}
\label{sec:background_quantum_game_theory}



In the article ``Quantum information approach to normal representation of extensive games''\cite{Fra2011a}, the authors propose a representation
for both normal and finite extensive form games\cite{Fra2011}. This definition is based on the premise that any strategic game can be represented as an extensive form game where all the players have no knowledge about the actions taken by other players. 

The representation assumes that there
are only two available actions, which can be represented by a qubit.
Those actions could be a yes/no decision or a cooperate/defeat as
those found in many classical game theory problems. 

A game in this
form is represented by a six-tuple\ref{eq:quantum_game_six_tuple},
where:

\begin{equation}
\Gamma=(\mathcal{H}^{2^{a}},\: N,\:\vert\psi_{in}\rangle,\:\xi,\:\{\mathcal{U}_{j}\},\:\{E_{i}\})\label{eq:quantum_game_six_tuple}
\end{equation}

\begin{itemize}
\item $a$ is the number of actions (qubits), in the game; 
\item $\mathcal{H}^{2^{a}}$ is a $2^{a}$-dimensional Hilbert space constructed
as $\otimes_{j=1}^{a}\mathbb{C}^{2}$, with basis $\mathcal{B}$;
\item $\vert\psi_{in}\rangle$ is the initial state of the compound-system
composed by $a$ qubits: $\vert\varphi_{1}\rangle,\:\vert\varphi_{2}\rangle, ..., \vert\varphi_{j}\rangle, ..., \vert\varphi_{a}\rangle$;
\item $\xi$ is a mapping function that assigns each action to a player;
\item For each qubit $j$ $\mathcal{U}_{j}$ is a subset of unitary operators from $\mathsf{SU}(2)$ (the general for the 2 dimensional Special Unitary Group is presented in Equation \ref{eq:general_unitary_special_one}).
These operators can be used by the player to manipulate her qubit(s);
\item Finally, for each player $i$, $E_{i}$ is a utility functional that
specifies her payoff. This is done by attributing a real number (representing a expected utility $ u_{i}(b)$, in \ref{eq:quantum_game_definition_payoff_func}), to the measurement for the projection of the final state (\ref{eq:quantum_game_definition_move}), on a basis from the $\mathcal{B}$\ref{eq:quantum_game_definition_payoff_func}.\end{itemize}

\begin{equation}
\begin{split}
\mathcal{U}_{j}(w,x,y,z)=w.I + ix.\sigma_{x} + iy.\sigma_{y} + iz.\sigma_{z}, \\  w,x,y,z \in \mathbb{R} \wedge  
w^2 + x^2 + y^2 + z^2 =1 
\end{split}
\label{eq:general_unitary_special_one}
\end{equation}

\begin{equation}
E_{i}=\sum_{b \in \mathcal{B}} u_{i}(b)\vert \langle b\vert \psi_{fin}\rangle\vert^{2}, u_{i}(b) \in \mathbb{R}
\label{eq:quantum_game_definition_payoff_func}
\end{equation}

The strategy of a player $i$ is a map $\tau_{i}$ which assigns a
unitary operator $U_{j}$ to every qubit $j$ that is manipulated
by the player (\inputencoding{latin1}{$j$$\in\xi^{-1}(i)$}\inputencoding{latin9}).
The simultaneous move is represented in \ref{eq:quantum_game_definition_move}.

\begin{equation}
\vert\psi_{fin}\rangle=\otimes_{i=1}^{N}\otimes_{j\in\xi^{-1}(i)}\mathcal{U}_{j}\vert\psi_{in}\rangle
\label{eq:quantum_game_definition_move}
\end{equation}

The tensor product of all the operators chosen by the players is referred as a super-operator, which act upon the game system.

In the article "Quantum Games and Quantum Strategies"\cite{Eisert2008}, the authors describe a quantization scheme for the Prisoner's Dilemma game. The way the initial state, $\vert\psi_{in}\rangle$, is set-up in the game system provide a way to entangle the game system, thus allowing the study of this phenomenon. For a $2$-player game, this is accomplished using Equation \ref{eq:2_4:estado_inicial_prisioneiro}. $\mathcal{J}$ is a matrix exponential that is chosen because it can commute with a super-operator, made from the subset of unitary operators that contains the identity matrix and the Bit-flip Pauli matrix\cite{citeulike:10961388}. According to \cite{Eisert2008}, the parameter $\gamma$ becomes a way to measure the entanglement in the system.

\begin{equation}
\label{eq:2_4:matrix_exponencial_esoterica}
\mathcal{J}=exp\left\{ i\frac{\gamma}{2}\left[\begin{array}{cc}
0 & 1\\
1 & 0
\end{array}\right]\otimes\left[\begin{array}{cc}
0 & 1\\
1 & 0
\end{array}\right]\right\}
\end{equation} 

\begin{equation}
\label{eq:2_4:estado_inicial_prisioneiro}
\begin{split}
\vert\psi_{in}(\gamma)\rangle=exp\left\{ i\frac{\gamma}{2}\left[\begin{array}{cc}
0 & 1\\
1 & 0
\end{array}\right]\otimes\left[\begin{array}{cc}
0 & 1\\
1 & 0
\end{array}\right]\right\} \vert00\rangle \\
=cos(\frac{\gamma}{2})\vert00\rangle+isin(\frac{\gamma}{2})\vert11\rangle,\gamma\in(0,\pi)
\end{split}
\end{equation}





