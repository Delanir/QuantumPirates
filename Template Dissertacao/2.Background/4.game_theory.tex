\section{Game Theory}
\label{sec:background:game_theory}

\subsection{Zero-Sum Game}
\label{subsec:background:game_theory_zero_sum_game}

A zero-sum game is a mathematical representation of a system where the gains are completely evened out by the losses; this means that the sum of the utilities of all players will always be zero.


\section{Quantum Game Theory}
\label{sec:background_quantum_game_theory}

[Intro SAUCEs]

The definition of Piotr Frackiewicz in Quantum information approach
to normal representation of extensive games proposes a representation
for finite extensive form games. The representation assumes that there
are only two available actions, which can be represented by a qubit.
Those actions could be a yes/no decision or a cooperate/defeat as
those found in many classical game theory problems. A game in this
form is represented by a six-tuple\ref{eq:quantum_game_six_tuple},
where:

\begin{equation}
\Gamma=(\mathcal{H}^{a},\: N,\:\vert\psi_{in}\rangle,\:\xi,\:\{\mathcal{U}_{j}\},\:\{E_{i}\})\label{eq:quantum_game_six_tuple}
\end{equation}

\begin{itemize}
\item $a$ is the number of actions (qubits), in the game; 
\item $\mathcal{H}^{a}$ is a $a$-dimensional Hilbert space constructed
as $\otimes_{j=1}^{a}\mathbb{C}^{2}$, with basis $\mathcal{B}$;
\item $\vert\psi_{in}\rangle$ is the initial state of the compound-system
composed by $a$ qubits: $\vert\varphi_{1}\rangle,\:\vert\varphi_{2}\rangle, ..., \vert\varphi_{j}\rangle, ..., \vert\varphi_{a}\rangle$;
\item $\xi$ is a mapping function that assigns each action to a player;
\item For each qubit $j$ $\mathcal{U}_{j}$ is a subset of unitary operators
that can be used by the player to manipulate her qubit;
\item Finally, for each player $i$ $E_{i}$ is a utility functional that
specifies her payoff. This is done by attributing a real numbered utility to the measurent of the projection of a basis in the final state\ref{eq:quantum_game_definition_payoff_func}.\end{itemize}

\begin{equation}
E_{i}=\sum_{b \in \mathcal{B}} u_{i}(b)\vert \langle b\vert \psi_{fin}\rangle\vert^{2}, u_{i}(b) \in \mathbb{R}
\label{eq:quantum_game_definition_payoff_func}
\end{equation}

The strategy of a player $i$ is a map $\tau_{i}$ which assigns a
unitary operator $U_{j}$ to every qubit $j$ that is manipulated
by the player (\inputencoding{latin1}{$j$$\in\xi^{-1}(i)$}\inputencoding{latin9}).
The simultaneous move is represented in \ref{eq:quantum_game_definition_move}.

\begin{equation}
\vert\psi_{fin}\rangle=\otimes_{i=1}^{N}\otimes_{j\in\xi^{-1}(i)}U_{j}\vert\psi_{in}\rangle\label{eq:quantum_game_definition_move}
\end{equation}
