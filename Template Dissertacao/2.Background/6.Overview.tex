\section{Overview}
\label{sec:background_overview}

In this chapter we provided a theoretical background that can help the reader to grasp the contents presented in the subsequent chapters.

We started by laying down a comparison between the classical probability theory and the quantum probability theory. The von Neumann probability is the mathematical foundation behind Quantum Mechanics. 

The von Neumann probability differs from the classical mainly because mutually exclusive events can interfere, this happens in the Double-slit Experiment, this means that the third axiom of Kolmogorov does not hold true in this probability theory. The concept of probability is deeply intertwined with Quantum Mechanics. 

In the Quantum Computing we presented the fundamental concepts. The book ```Quantum Computing - A Gentle Introduction''\cite{Rieffel2011} provides a more in-depth resource on this subject; including the Shor's algorithm that is used to find prime factorizations, and the Deutsch-Jozsa algorithm, that with a single query decides if an unknown function is constant or balanced. These algorithms fall outside the scope of this document.

The Game Theory section contains a brief description of some fundamental concepts from Game Theory needed in order to understand this work. A comprehensive reference such as \cite{Osborne2004} can be consulted for a deeper insight on the subject.

%In the Quantum Game Theory we focused on the definition of Quantum Game deemed more relevant for this work. Other approaches such as 

