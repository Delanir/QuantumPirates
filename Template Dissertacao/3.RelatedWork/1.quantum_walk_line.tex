\section{Quantum Walks}
\label{sec:quantum_walk}

\subsection{Quantum Walk on a Line}
\label{subsec:quantum_walk_line}

A Quantum Walk is the quantum version of random walks, which are a mathematical formalism to describe a path constructed by random steps. This processes can be described by a Markov Chain. 

We can define the Discrete Quantum Walk on a Line as a series of Left/Right decisions. Understanding this algorithm is important towards being able to define and design more complex algorithms.
 
We followed an approach suggested by \cite{Ambainis} towards simulating n-steps of a Quantum Walk on a Line, the algorithm can be consulted on \ref{ap:a}.
In a discrete quantum walk in a line we want to preserve some properties, namely: the probability of turning left should be equal to the probability of turning right.  To represent a state in this algorithm we will need the number of the node and a direction (identified as L,R) \ref{eq:3_qwl_state}.

\begin{equation}
\label{eq:3_qwl_state}
\vert \psi\rangle = \vert n, L\rangle
\end{equation}

Having two equally possible choices we can resort to a coin metaphor\cite{Ambainis}\cite{Ambainis2008}. We toss a coin and go either Left or Right depending on the result. In a quantum version we need to define a Coin Operator (Coin Matrix) which is responsible to imprint a direction to our state with equal probability. This operator is a unitary matrix in a 2-dimension Hilbert space. Some examples of Coin Operators are the Hadamard matrix\ref{eq:hadamard} and a symmetric unitary matrix \ref{eq:qwl_symmetric}.

\begin{equation}
\label{eq:hadamard}
H=\frac{1}{\sqrt{2}}\left[\begin{array}{cc}
1 & 1\\
1 & -1
\end{array}\right]
\end{equation}

\begin{equation}
\label{eq:qwl_symmetric}
\left[\begin{array}{cc}
\frac{1}{\sqrt{2}} & \frac{i}{\sqrt{2}}\\
\frac{i}{\sqrt{2}} & \frac{1}{\sqrt{2}}
\end{array}\right]
\end{equation}

Taking the Hadamard matrix as an example\ref{eq:hadamard}, the coin matrix will operate on the state in the following way \ref{eq:qwl_1}\ref{eq:qwl_2}\cite{Ambainis}.

\begin{equation}
\label{eq:qwl_1}
C\vert n, L\rangle = \frac{1}{\sqrt{2}} \vert n, L\rangle + \frac{1}{\sqrt{2}} \vert n, R\rangle
\end{equation}
  
\begin{equation}
\label{eq:qwl_2}
C\vert n, R\rangle = \frac{1}{\sqrt{2}} \vert n, L\rangle - \frac{1}{\sqrt{2}} \vert n, R\rangle
\end{equation} 

The Coin Matrix obtains its name by being the quantum equivalent of flipping a classic coin. After tossing a coin comes an operator that will move the node in the direction assigned. The operator responsible for this modification is commonly referred as Shift Operator \ref{eq:qwl_3}\ref{eq:qwl_4}. 

\begin{equation}
\label{eq:qwl_3}
S\vert n, L\rangle = \frac{1}{\sqrt{2}} \vert n-1, L\rangle
\end{equation}

\begin{equation}
\label{eq:qwl_4}
S\vert n, R\rangle = \frac{1}{\sqrt{2}} \vert n+1, R\rangle
\end{equation} 

These matrices (Coin Matrix and Shift Operator) are referred conceptually ubiquitously throughout the literature [sources], therefore it is important to be familiar with them. A single step of the algorithm \ref{ap:a} is illustrated in Figure \ref{fig:qwl_tree}.

miau miau miau miau miau miau miau miau miau miau miau miau miau miau miau miau miau miau miau miau miau miau miau miau miau miau miau miau miau miau miau miau miau miau miau miau miau miau miau miau miau miau miau miau miau miau miau miau miau miau miau miau miau miau miau miau miau miau miau miau miau miau miau miau miau 

\begin{figure}[h]
\centering 

\includegraphics[scale=0.50]{Figures/quantum_walk_line.png}
\caption{Simulating a step of a discrete quantum walk on a line. In the beginning we have a state characterized by the position $(0)$ and a direction (either Left or Right).}
\label{fig:qwl_tree}
\end{figure}


Depending on the Coin Matrix we can get different distributions. In Figures \ref{fig:qwl_hadamard} and \ref{fig:qwl_symetric}

\begin{figure}[h]
\centering 
\includegraphics[scale=0.50]{Figures/quantum_walk_line_symetric.png}
\caption{30 Step of the Simulation \ref{ap:a} using Matrix \ref{eq:hadamard} as a Coin Operator.}
\label{fig:qwl_symetric}
\end{figure}

\begin{figure}[h]
\centering 
\includegraphics[scale=0.50]{Figures/quantum_walk_line_hadamard.png}
\caption{30 Step of the Simulation \ref{ap:a} using a Hadamard Matrix \ref{eq:hadamard} as a Coin Operator.}
\label{fig:qwl_hadamard}
\end{figure}

Starting on the middle of a line we can shift one unit left or right.
If we took the classical approach in which we tossed a fair coin, and after n-steps we measured the final node repeatedly, by the \ac{CLT} the final distribution would converge to a normal distribution.
