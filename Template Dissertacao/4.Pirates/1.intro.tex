\section{Pirate Game}
\label{sec:pirate}

\begin{emph}
On selecting the problem...
\end{emph}


\subsection{Problem Description}
\label{subsec:description}

The Pirate Game is a multiplayer version of the Ultimatum game that is usually stated as follows:

\begin{quotation}
Suppose there are 5 rational pirates: A; B; C; D; E. The pirates have a  loot of 100 gold coins to divide among themselves.


As the pirates have a strict hierarchy, in which pirate A is the captain and E has the lowest rank, the highest ranking pirate alive will propose a division. Then each pirate will cast a vote on whether or not to accept the proposal. 

If a majority or a tie is reached the goods will be allocated according to the proposal. Otherwise the proposer will be thrown overboard and the next pirate in the hierarchy assumes the place of the captain. 

We consider that each pirate privileges her survival, and then will want to maximize the number of coins received. When the result is indifferent the pirates prefer to throw another pirate overboard and thus climbing in the hierarchy. 
\end{quotation}

We can arrive at an equilibrium in this problem by using backward induction. 


\subsection{Quantum considerations}
\label{subsec:description_2}

In order to model the mencioned problem it's necessary to captura its essence.
In terms or mechanics and steps, this problem could be described using 3 players and later extended to any number $N$ of players. The highest ranking pirate in the hierachy will be responsible to make a proposal. This proposal can be modelled as describing the payoff functionals for every pirate, according to some rules. This goods allocation proposal will be executed if there is a majority (or a tie), in the voting step. In the voting step, each pirate will cast a vote by choosing simultaneously a operator (that can mean either accep or reject the proposal).
''''''Depending on the measurement outcome that occurs with probability x, the players will act on the second round. The Luders rule is applied here. The highest ranking player will no longer have access to the YES or NO operatos, instead he will only be able to manipulate de system using a symetric matrix like a coin operator, in order to not affect the voting.

Keeping the system in a higher dimension is considered because of the following reasons: We don't have proof that the states are separable. There are various mappings that fit the system composition. 


Initial state: variable to study.
Payoff functions: variable to study.

\begin{emph}
One important aspect is that the payoff functional changes
\end{emph}