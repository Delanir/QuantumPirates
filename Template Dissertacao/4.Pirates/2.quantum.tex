
\section{Quantum Pirate Game}
\label{sec:quantum_pirate}

The original Pirate Game is posed from the point of view of the captain. How should she allocate the treasure to the crew in order to maximize her payoff.
We can find the a equilibrium to the original Pirates Game and, while the solution may seem unexpected at first sight, it is fully described using backwards induction. 

When modelling this problem from a quantum theory perspective we are faced with some questions, such as:

\begin{itemize}
\item Will the initial conditions provide different equilibria? 

\item What are the similarities with the classical problem? 

\item Is it possible for a captain, in a situation where we have more than two pirates left, to acquire all the coins?

\end{itemize} 

The main difference from the original problem will rely on how the system is set up. 
We propose to study this problem for a $3$ player game and trying to extrapolate for $N$ players. 

We will analyse the role of entanglement and superposition in the game system. 

Another aspect worth studying is the variation in the coin distribution on the payoff functions for the players. We are particularly interested in studying the classical equilibrium where the captain retains $99$ coins and gives a single coin to the player with the lowest rank. Moreover we want to study what happens when the captain tries to get all the coins.



\subsection{Quantum Model}
\label{subsec:description_2}

In order to model the problem we will start by defining it using the definition of quantum game ($\Gamma$), referred in \ref{eq:quantum_game_six_tuple}, Section \ref{sec:background_quantum_game_theory}\cite{Fra2011a}.

We want to keep the problem as close to the original as possible in order to better compare the results. Thus we will analyse the game from the point of view of the captain. Will her best response change?

For the purpose of demonstration this problem could be described using $3$ players; the lowest number of players that has an equilibrium in which the captain has to bribe another pirate. 

We begin by assigning an offset to each pirate (in order to identify her), as in the Section \ref{subsec:description}. The captain is number $1$ and the lower the number the higher the rank. 



\subsubsection{Game system: Setting up the Initial State}
\label{subsec:pirates_initialstate}

A game $\Gamma$ can be viewed as a system composed by qubits manipulated by players. In a $3$ player game there will be $3$ qubits, each manipulated by a different player. 

With $3$ players our system with be represented in a $\mathcal{H}^{8}$ using a state $\psi$. This means that to represent our system we will need $2^{3}\times 1$ vectors, our system grows exponentially with the number of players/qubits. Each pure basis $\mathcal{B}= \{ \vert 000\rangle , \vert 001\rangle , \vert 010\rangle , \vert 011\rangle , \vert 100\rangle , \vert 101\rangle , \vert 110\rangle , \vert 111\rangle \}$ of $\mathcal{H}^{8}$ will represent a possible outcome in the game. We assign a pure basis as $\vert 0\rangle = \vert C\rangle$ (``C'' from ``Cooperate''), and $\vert 1\rangle = \vert D\rangle$ (``D'' from ``Defect'').

For 200 players (for example), this game would be impractical to simulate in a classical computer. In this regard a quantum computer may enhance our power to simulate this kinds of experiments \cite{Rieffel2011}.

\begin{figure}[h]
\centering 
\includegraphics[scale=0.55]{Figures/architecture/GameTree/Slide1.png}
\caption{Game tree representation of the operator possibilities for a $3$-player game. Red circles represent failed proposals, green represent accepted proposals. }
\label{fig:pg_architecturegametree}
\end{figure}

The initial system ($\vert \psi_{0}(\gamma) \rangle$), will be set up by defining an entanglement coefficient $\gamma$, that affect the way the three qubits (belonging to the three pirate players), are related; this is shown in Equation \ref{eq:estado_inicial_pg}. 
We will entangle our state by applying the gate $\mathcal{J}$ \cite{Letters2002}. The parameter $\gamma$ becomes a way to measure the entanglement in the system\cite{Eisert2008}. 

The concept of entanglement is crucial to explain some phenomena in Quantum Mechanics (Section \ref{subsec:entanglement}).We analysed the role of the entanglement of the system since other examples researched pointed to it being the proeminent factor regarding behaviour changes from the classsical perspective\cite{Fra2011a}\cite{Fra2011}\cite{Letters2002}\cite{Khan2011}\cite{Ricketts2006}. 

We can interpret the existence (or non-existence), of entanglement or superposition in the initial system as an unbreakable contract between the players\cite{Piotrowski}. The initial state starts by revealing a group of pirates that cooperate by default. We chose this initial set-up because it is prevalent in the literature, and we want to test if there is any equilibrium situation where the first captain can pass her proposal while taking all the 100 coins. 


The index $0$ represents the depth of the game tree which can be examined in Figure \ref{fig:pg_architecturegametree}.

Due to the nature of quantum mechanics we have to pay attention of how we set-up our architecture; we cannot copy or clone unknown quantum states (No-cloning Theorem)\cite{Rieffel2011}. 



\begin{equation}
\mathcal{J}=exp\left\{ i\frac{\gamma}{2}\left[\begin{array}{cc}
0 & 1\\
1 & 0
\end{array}\right]\otimes\left[\begin{array}{cc}
0 & 1\\
1 & 0
\end{array}\right]
\otimes\left[\begin{array}{cc}
0 & 1\\
1 & 0
\end{array}\right]
\right\}
\label{eq:matrix_exponencial_esoterica}
\end{equation} 

\begin{equation}
%\vert \psi_{0}(\gamma) \rangle= cos( \frac{\gamma}{2})\vert 00\rangle+ isin(\frac{\gamma}{2})\vert 11 \rangle, \gamma \in (0,\pi)
\begin{split}
\vert\psi_{in}(\gamma)\rangle=exp\left\{ i\frac{\gamma}{2}\left[\begin{array}{cc}
0 & 1\\
1 & 0
\end{array}\right]\otimes\left[\begin{array}{cc}
0 & 1\\
1 & 0
\end{array}\right]\otimes\left[\begin{array}{cc}
0 & 1\\
1 & 0
\end{array}\right]\right\} \vert000\rangle \\
=cos(\frac{\gamma}{2})\vert000\rangle+isin(\frac{\gamma}{2})\vert111\rangle,\gamma\in(0,\pi)
\end{split}
\label{eq:estado_inicial_pg}
\end{equation}



\subsubsection{Strategic Space}
\label{subsec:strategic_space}

In Equation \ref{eq:quantum_game_six_tuple} there is the notion of a subset of unitary operators that the players can use to manipulate their assigned qubits. 

Each player will be able to manipulate a qubit in the system, in this case $\vert\varphi_{1}\rangle,\:\vert\varphi_{2}\rangle,$ and $\vert\varphi_{3}\rangle$, with one of two operators shown in Equation \ref{eq:operators_piratas_quanticos}. 

An operator is an unitary $2\times2$ matrix that is used to manipulate a qubit in the system.
This restriction of the strategic space is relevant to keep the problem as close to the classical version as possible. The two operators will correspond to the action of voting ``Yes'' or to Cooperate, and voting ``No'', meaning that they will not accept the proposal.  

The cooperation operator will be represented by the Identity operator ($o_{i0}$, where i identifies the qubit upon which player i will act). When assigned to a qubit this operator will leave it unchanged. 

The defection operator ($D$), will be represented by one of Pauli's Operators - the Bit-flip operator. This operator was chosen because it performs the classical operation NOT on a qubit.

These operators are also permutation matrices, so our players are in fact permuting the state of their qubit as in the roulette quantum model (Section \ref{subsec:quantum_roulette}). It is also noteworthy that this operators correspond to pure-strategies.

\begin{equation}
\label{eq:operators_piratas_quanticos}
\mathcal{U}_{i} = \begin{cases}
C = o_{i0}=\left[\begin{array}{cc}
1 & 0\\
0 & 1
\end{array}\right]\\
D = o_{i1}=\left[\begin{array}{cc}
0 & 1\\
1 & 0
\end{array}\right]
\end{cases} , i \in \{ 1, 2, 3 \}
\end{equation}

When a player is no longer able to cast a vote she will play a symmetric coin matrix represented in Equation \ref{eq:operators_piratas_quanticos_esoterico}. Like in the example of the Quantum walk in a line, described in \ref{sec:quantum_walk_line}, the symmetric coins matrix allows to explore both choices for operators (Cooperate/Defect).

\begin{equation}
H = \left[\begin{array}{cc}
\frac{1}{\sqrt{2}} & \frac{i}{\sqrt{2}}\\
\frac{i}{\sqrt{2}} & \frac{1}{\sqrt{2}}
\end{array}\right]
\label{eq:operators_piratas_quanticos_esoterico}
\end{equation}

In Quantization schemas of the Prisoner's Dilemma\cite{Letters2002}\cite{Eisert2008}, and Quantum Ultimatum Game\cite{Fra2011}, the strategic space, described in Equation \ref{eq:operadoresinfinitus}, was analysed allowed a infinity of mixed quantum strategies. In the Quantum Roulette Game\cite{Salimi2009} and \cite{Meyer1999} we have a demonstration that in a classical two-person zero-sum strategic game, if one person adopts a quantum strategy, she has a better chance of winning the game. 

However we claim that allowing mixed quantum strategies would degenerate the voting problem. In the original problem the stages of the game are dependent of the actions taken by the players. Also studying the strategic space in the game system (represented in a $\mathcal{H}^{8}$), would be a daunting process in terms of simulation. 

\begin{equation}
\mathcal{U}(\theta,\phi) = \left[\begin{array}{cc}
cos(\frac{\phi}{2}) & e^{i\phi}sin(\frac{\phi}{2})\\
-e^{-i\phi}sin(\frac{\phi}{2}) & cos(\frac{\phi}{2})
\end{array}\right]
\label{eq:operadoresinfinitus}
\end{equation}

\subsubsection{Final State}
\label{subsec:pirates_finalstate}

We can play the Pirate Game by considering a succession of steps or voting rounds. In each step we have a simultaneous move(the players sellect their strategies at the same time), however, considering the potential rounds the game has, we have a sequential game. 

With three players, the first move will correspond to the player 1 (or the captain), if the proposal fails we will proceed to the second step in the game, where the remaining two players will vote on a new proposal made by player 2 (who will be the new captain). The first captain is indifferent to the outcome of the second game so he uses a coin matrix to vote.

After a move we have a quantum state that can be identified by an index which points to the depth $k$ of the game, $\psi_{k}$ ( Figure \ref{fig:pg_architecturegametree}). This state is calculated by constructing a super-operator, by performing the tensor product of each player chosen strategy, as shown in Equation \ref{eq:operators_piratas_quanticos}. The super-operator, containing each player's strategy, will then be applied to the initial state, this will correspond to the players making a simultaneous move represented in Equation\ref{eq:piratas_final_move}.

In the Figure \ref{fig:pg_architecture3players} we have a representation of the game.We start by building our initial state $k-1$, then the players  select their strategies, a super operator is constructed by performing a tensor product of the selected operators. In order to calculate the expected payoff functions we need to de-entangle the system, before measuring. The act of measuring, in quantum computing, gives an expected value that can be understood as the probability of the system collapsing into that state. 

We can de-entangle the our $\mathcal{H}^{8}$ system by applying $\mathcal{J}^{\dagger}$ (Equation \ref{eq:piratas_final_move2}), this will produce a final state that we will be able to measure. If we do not apply the inverse transformation $\mathcal{J}^{\dagger}$ we are introducing errors in the system (when the entanglement parameter $\gamma$ is different than $0$), because we are introducing a correlation between the qubits in the system. In the Figure \ref{fig:pg_architecture3players} we have represented the way we entangle and de-entangle the system.

\begin{equation}
\vert\psi_{k}\rangle=\otimes_{i=1}^{N} \mathcal{U}_{i}\vert\psi_{k-1}\rangle
\label{eq:piratas_final_move}
\end{equation}

\begin{equation}
\vert\psi_{fin}\rangle= \mathcal{J}^{\dagger}\vert\psi_{k}\rangle
\label{eq:piratas_final_move2}
\end{equation}

\begin{figure}[h]
\centering 
\includegraphics[scale=0.35]{Figures/architecture/esquema/esquema.png}
\caption{Scheme that represents the set-up of the $3$-player Pirate Game. Before we measure the final result we need to apply the transpose operator $\mathcal{J}^{\dagger}$. }
\label{fig:pg_architecture3players}
\end{figure}



In the original game the voting results are displayed between rounds, thus being an extensive form game. In the article ``Quantum information approach to the ultimatum game''\cite{Fra2011}, the authors model an extensive form game, and claim that it is more natural measuring the system between stages of the game (from a quantum perspective), instead of observing the actions taken by the players. However with our set-up measuring between rounds will destroy the initial entanglement which means that the effects of entanglement will only be present in the first stage of the game. 

This leads to the decision that to quantize the stages in the game we should be observing the actions taken by the players. The actions taken by the players will indicate whether the game stops (if the proposal is accepted), or if it ensues to a new stage (Figure \ref{fig:pg_architecture3players_architecture}). 

In order to preserve the consider the phenomenon of entanglement in the subgames, while measuring the state after a stage (applying $J^{\dagger}$), methodology suggested by \cite{Fra2011} we would need to reapply an entanglement gate ($J_{2}^{\dagger}$), with a new entanglement parameter $\gamma_{2}$. This would introduce a new parameter introducing a new layer of complexity to the simulation; for each initial entanglement coeficient $\gamma$ we would have an infinity of entanglement coeficients for each sub-game. When the two entanglement coefitients are equal the probability distributions associated with each pure-base of the system the are equal to our approach of observing the actions taken by the players.

\begin{figure}[h]
\centering 
\includegraphics[scale=0.50]{Figures/architecture/esquema/Slide5.png}
\caption{The diagram represents the simulation of the Quantum Pirate Game. }
\label{fig:pg_architecture3players_architecture}
\end{figure}

The next pirate in the hierarchy will then become the captain, and the previous captains will no longer be able to cast a vote (a symmetric Coin matrix will be used instead). 
Unlike the classic version, where the captain destined to be killed are tossed aside, in the quantum version they will be only thrown off board when we measure the final state and their expected utility is negative.
In this stage we must take into consideration that the players are pronouncing their votes taking into account their previous actions.




\subsubsection{Utility}
\label{subsec:pirates_utility}

To build the expected payoff functionals for the three player situation we must take into account the sub-games created when the proposal is rejected. In Figure \ref{fig:pg_architecturegametree} we can see an extensive form representation of the game.

As defined on Equation \ref{eq:quantum_game_definition_payoff_func},for each player we must specify a utility functional that attributes a real number to the measurement of the projection of a basis in the quantum state that we get after the game. 


This measurement can be understood as a probability of the system collapsing into that state (that derives from the Born Rule, Section \ref{subsubsec:bornrule}).


These utility functions will represent the degree of satisfaction for each pirate after game by attributing a real number to a measurement performed to the system (as in Equation \ref{eq:quantum_game_definition_payoff_func}, Section \ref{sec:background_quantum_game_theory}). 
The real numbers used convey the logical relations of utility posed by the original problem description. Those numbers will represent the utility associated with the number of coins that a pirate gets, a death penalty, and a small incentive to climb the hierarchy. As each pirate wants to maximize her utility, the Nash equilibrium will be thoroughly used to find the strategies that the pirates will adopt\cite{nash50}\cite{Nash51}.

The number of coins will translate directly the utility associated with getting those coins. For example if a pirate receives 5 gold coins and the proposal is accepted he will get a utility of 5. 

The highest ranking pirate in the hierarchy will be responsible to make a proposal to divide the 100 gold coins. This proposal is modelled as choosing some parameters for the payoff functionals for every player, according to some rules. For the initial step in the game with three pirates these parameters will be $\alpha_{1}, \alpha_{2}, \alpha_{3}$, and they will obey to the Equation \ref{eq:goodss}, where $k$ is the offset of the current captain, and $N$ the number of pirates in the game. 

\begin{equation}
\label{eq:goodss}
\sum_{i=k}^{N}\alpha_{i}=100, \forall i :\alpha_{i}\in\mathbb{N}_{0}
\end{equation}

The most interesting values for $(\alpha_{1}, \alpha_{2}, \alpha_{3})$ will be the allocation that results in a Nash equilibrium in the original Pirate Game $(99, 0, 1)$, and the case where the captain maximizes the number of coins he can get $(100, 0, 0)$. Will the game modelled as a quantum system allow the captain to acquire all the coins?

The proposed goods allocation will be executed if there is a majority (or a tie), in the voting step. A step in the game consists on the highest ranking pirate defining a proposal and the subsequent vote, where all players choose simultaneously an operator. 

If the proposal is rejected the captain will be thrown off board, to account for the fact that this situation is very undesirable for the captain he will receive a negative payoff of $-200$. This value was derived from the fact that a pirate values her integrity more than any number of coins she might receive.

%chosen to be much less than the highest number of coins a pirate could get.



\begin{quotation}
``When the result is indifferent the pirates prefer to throw another pirate overboard and thus climbing in the hierarchy.''
\end{quotation}

This means that the pirates have a small incentive to climb the hierarchy. For example in the three player classical game, the third player, who has the lowest rank, will prefer to defect the initial proposal if the player 1 doesn't give her a coin, even knowing that in the second round the player 2 will be able to keep the 100 coins. We will account for this preference by assigning an expected value of half a coin ($0.5$), to the payoff of the players that will climb on the hierarchy if the voting fails.


We can observe that in Equations \ref{eq:pirates_payoff32} and \ref{eq:pirates_payoff3} that we have two separate groups of outcomes: 
\begin{itemize}
\item Outcomes where the proposal is passed:
\begin{itemize}
\item $\vert CCC\rangle$ or $\vert000\rangle$ (which we measure with the operation $\vert\langle000\vert\psi_{fin}\rangle\vert^{2}$ );
\item $\vert DCC\rangle$ or $\vert100\rangle$;
\item $\vert CDC\rangle$ or $\vert010\rangle$;
\item $\vert CCD\rangle$ or $\vert001\rangle$.
\end{itemize}
\item outcomes where the captain will be eliminated and the remaining players will keep playing:
\begin{itemize}
\item $\vert DDD\rangle$ or $\vert111\rangle$;
\item $\vert DDC\rangle$ or $\vert110\rangle$;
\item $\vert CDD\rangle$ or $\vert011\rangle$;
\item $\vert DCD\rangle$ or $\vert101\rangle$.
\end{itemize}
\end{itemize}



The final payoff function (for example Equation \ref{eq:pirates_payoff32} for a $3$ player game), will be calculated recursively, the base case being the $2$ player sub-game in a $3$ player system will be Equation \ref{eq:pirates_payoff3}.

 

 \begin{equation}
 \begin{cases}
\begin{split}
E_{11}(\vert\psi_{fin}\rangle, \alpha_{1})=\alpha_{1}\times(\vert\langle000\vert\psi_{fin}\rangle\vert^{2} + \vert\langle100\vert\psi_{fin}\rangle\vert^{2}
+ \vert\langle010\vert\psi_{fin}\rangle\vert^{2}
+ \vert\langle001\vert\psi_{fin}\rangle\vert^{2}
 ) - \\
 - 200\times(\vert\langle111\vert\psi_{fin}\rangle\vert^{2} + \vert\langle110\vert\psi_{fin}\rangle\vert^{2}
+ \vert\langle101\vert\psi_{fin}\rangle\vert^{2}
+ \vert\langle011\vert\psi_{fin}\rangle\vert^{2}
 )
\end{split}
\\
\begin{split}
E_{12}(\vert\psi_{fin}\rangle, \alpha_{2})=\alpha_{2}\times(\vert\langle000\vert\psi_{fin}\rangle\vert^{2} + \vert\langle100\vert\psi_{fin}\rangle\vert^{2}
+ \vert\langle010\vert\psi_{fin}\rangle\vert^{2}
+ \vert\langle001\vert\psi_{fin}\rangle\vert^{2}
 ) - \\
 + (0.5 + E_{22})\times(\vert\langle111\vert\psi_{fin}\rangle\vert^{2} + \vert\langle110\vert\psi_{fin}\rangle\vert^{2}
+ \vert\langle101\vert\psi_{fin}\rangle\vert^{2}
+ \vert\langle011\vert\psi_{fin}\rangle\vert^{2}
 )
\end{split}
\\
\begin{split}
E_{13}(\vert\psi_{fin}\rangle, \alpha_{3})=\alpha_{3}\times(\vert\langle000\vert\psi_{fin}\rangle\vert^{2} + \vert\langle100\vert\psi_{fin}\rangle\vert^{2}
+ \vert\langle010\vert\psi_{fin}\rangle\vert^{2}
+ \vert\langle001\vert\psi_{fin}\rangle\vert^{2}
 ) - \\
 + (0.5 + E_{23})\times(\vert\langle111\vert\psi_{fin}\rangle\vert^{2} + \vert\langle110\vert\psi_{fin}\rangle\vert^{2}
+ \vert\langle101\vert\psi_{fin}\rangle\vert^{2}
+ \vert\langle011\vert\psi_{fin}\rangle\vert^{2}
 )
\end{split}
\end{cases}
\label{eq:pirates_payoff32}
%\caption{Payoff funcionals for the $3$ player pirate game. The payoff function for the non-captain players is recursive because their decision to approve or reject the initial proposal will depend on how much they expect to gain in the next round ($E_{22}$ and $E_{23}$).}
\end{equation}

 \begin{equation}
\begin{cases}
\begin{split}
E_{22}(\vert\psi_{fin2}\rangle, \alpha_{2})=\alpha_{2}\times(\vert\langle000\vert\psi_{fin1}\rangle\vert^{2} + \vert\langle100\vert\psi_{fin1}\rangle\vert^{2}
+ \vert\langle010\vert\psi_{fin1}\rangle\vert^{2}
+ \vert\langle001\vert\psi_{fin1}\rangle\vert^{2}
 ) - \\
 + 0.5\times(\vert\langle111\vert\psi_{fin1}\rangle\vert^{2} + \vert\langle110\vert\psi_{fin1}\rangle\vert^{2}
+ \vert\langle101\vert\psi_{fin1}\rangle\vert^{2}
+ \vert\langle011\vert\psi_{fin1}\rangle\vert^{2}
 )
\end{split}
\\
\begin{split}
E_{23}(\vert\psi_{fin2}\rangle, \alpha_{3})=\alpha_{3}\times(\vert\langle000\vert\psi_{fin1}\rangle\vert^{2} + \vert\langle100\vert\psi_{fin1}\rangle\vert^{2}
+ \vert\langle010\vert\psi_{fin1}\rangle\vert^{2}
+ \vert\langle001\vert\psi_{fin1}\rangle\vert^{2}
 ) - \\
 + 100.5 \times(\vert\langle111\vert\psi_{fin1}\rangle\vert^{2} + \vert\langle110\vert\psi_{fin1}\rangle\vert^{2}
+ \vert\langle101\vert\psi_{fin1}\rangle\vert^{2}
+ \vert\langle011\vert\psi_{fin1}\rangle\vert^{2}
 )
\end{split}
\end{cases}
\label{eq:pirates_payoff3}
%\caption{Second round of a $3$ player game.The $2$ player subgame is the base case for this problem because one vote is enough to pass the proposal.}
\end{equation}

