\section{Quantum Pirate Game}
\label{sec:quantum_pirate}

\subsection{Hypothesis}
\label{subsec:qhipothesis}

The original Pirate Game is posed from the point of view of the captain. How should she allocate the treasure to the crew in order to maximize her payoff.
We can find the a equilibrium to the original Pirates Game and, while the solution may seem unexpected at first sight, it is fully described using backwards induction. 

When modelling this problem from a quantum theory perspective we are faced with some questions. The main difference from the original problem will rely on how the system is set up. Will the initial conditions provide different equilibria? Is there a condition where we are left with the classical problem? Is it possible for a captain, in a situation where we have more than two pirates left, to acquire all the coins?

We propose to study this problem for the $2$ and $3$ player games and trying to extrapolate for $N$ players. We will analyse the role of entanglement and superposition in the game system. Another aspect worth studying is the variation in the coin distribution on the payoff functions for the players.



\subsection{Quantum Model}
\label{subsec:description_2}

In order to model the problem we will start by defining it using the definition of quantum game ($\Gamma$), referred in \ref{eq:quantum_game_six_tuple}, Section \ref{sec:background_quantum_game_theory}\cite{Fra2011a}.

We want to keep the problem as close to the original as possible in order to better compare the results. Thus we will analyse the game from the point of view of the captain. Will her best response change?

In terms of mechanics and steps, this problem could be described using $3$ players and later extended to any number $N$ of players. 

We begin by assigning an offset to each pirate (in order to identify her), as in the Section \label{subsec:description}. The captain is number $1$ and the lower the number the higher the rank. 


\subsubsection{Strategic Space}
\label{subsec:strategic_space}

In Equation \ref{eq:quantum_game_six_tuple} there is the notion of a subset of unitary operators that the players can use to manipulate their assigned qubits. 

Each player will be able to manipulate a qubit in the system, in this case $\vert\varphi_{1}\rangle,\:\vert\varphi_{2}\rangle,$ and $\vert\varphi_{3}\rangle$, with one of two operators shown in Equation \ref{eq:operators_piratas_quanticos}. 

An operator is an unitary $2\times2$ matrix that is used to manipulate a qubit in the system.
This restriction of the strategic space is relevant to keep the problem as close to the classical version as possible. The two operators will correspond to the action of voting ``Yes'' or to Cooperate, and voting ``No'', meaning that they will not accept the proposal.  

The cooperation operator will be represented by the Identity operator ($o_{i0}$, where i identifies the qubit upon which player i will act). When assigned to a qubit this operator will leave it unchanged. 

The defection operator ($D$), will be represented by one of Pauli's Operators - the Bit-flip operator. This operator was chosen because it performs the classical operation NOT on a qubit.

These operators are also permutation matrices, so our players are in fact permutating the state of their qubit as in the roullete quantum model (Section \ref{subsec:quantum_roulette}).

\begin{equation}
\label{eq:operators_piratas_quanticos}
\mathcal{U}_{i} = \begin{cases}
C = o_{i0}=\left[\begin{array}{cc}
1 & 0\\
0 & 1
\end{array}\right]\\
D = o_{i1}=\left[\begin{array}{cc}
0 & 1\\
1 & 0
\end{array}\right]
\end{cases} , i \in \{ 1, 2, 3 \}
\end{equation}


\subsubsection{Game system: Setting up the Initial State}
\label{subsec:pirates_initialstate}

A game $\Gamma$ can be viewed as a system composed by qubits manipulated by players. In a $3$ player game there will be $3$ qubits, each manipulated by a different player. 

With $3$ players our system with be represented in a $\mathcal{H}^{3}$ using a state $\psi$. This means that to represent our system we will need vectors $2^{3}\times 1$ vectors, our system grows exponentialy with the number of players/qubits. This makes que a game for 200 players (for example), impractical to simulate in a classical computer. In this regard a quantum computer may enhance our power to simulate this kinds of experiments \ref{Rieffel2011}.


The initial system ($\vert \psi_{0}(\gamma) \rangle$), will be set up by defining an entanglement coefficient $\gamma$, that affect the way the three qubits (belonging to the three pirate players), are related \ref{eq:estado_inicial_pg}. The index $0$ represents the depth of the game tree which can be examined in Figure \ref{fig:pg_architecturegametree}.


The concept of entanglement is crucial to explain some phenomena in Quantum Mechanics (Section \ref{subsec:entanglement}). Analysing the role of entanglement in the system is also considered to be a major source of different results from the traditional Game Theory approach\cite{Fra2011a}\cite{Fra2011}\cite{Letters2002}\cite{Khan2011}\cite{Ricketts2006}. 
 
\begin{equation}
\label{eq:estado_inicial_pg}
\vert \psi_{0}(\gamma) \rangle= cos(\frac{ \gamma}{2})\vert 000\rangle+ isen(\frac{\gamma}{2})\vert 111 \rangle, \gamma \in (0,\pi)
\end{equation}



\subsubsection{Final State}
\label{subsec:pirates_finalstate}

We can play the Pirate Game by considering a succession of steps or voting rounds. In each step we have a simultaneous move, however considering the potential rounds the game has we have a sequential game. 

With three players, the first move will correspond to the player 1 (or the captain), if the proposal fails we will proceed to the second step in the game, where the remaining two players will vote on a new proposal made by player 2 (who will be the new captain). 

After a move we have a ``final state'' that can be identified by an index which points to the depth $k$ of the game ( Figure \ref{fig:pg_architecturegametree}). This state is calculated by constructing a super-operator, by performing the tensor product of each player chosen strategy\ref{eq:operators_piratas_quanticos}. The super-operator, containing each player strategy, will then be applied to the initial state, this will correspond to the players making a simultaneous move\ref{eq:piratas_final_move}.

In the Figure \ref{fig:pg_architecture3players} we have a step in the game.We start by building our initial state, then the players will select their strategy, a super operator is constructed by performing a tensor product of the selected operators.

\begin{equation}
\vert\psi_{k}\rangle=\otimes_{i=1}^{N} \mathcal{U}_{i}\vert\psi_{k-1}\rangle\label{eq:piratas_final_move}
\end{equation}

\begin{figure}[h]
\centering 
\includegraphics[scale=0.35]{Figures/architecture/esquema/Slide1.png}
\caption{Playing the first round of the Pirate Game with 3 players. }
\label{fig:pg_architecture3players}
\end{figure}

\begin{figure}[h]
\centering 
\includegraphics[scale=0.35]{Figures/architecture/esquema/Slide2.png}
\caption{Playing the second round of the Pirate Game with 3 players by performing an intermediate measuring. }
\label{fig:pg_architecture3players_2measure}
\end{figure}

\begin{figure}[h]
\centering 
\includegraphics[scale=0.35]{Figures/architecture/esquema/Slide3.png}
\caption{Playing the second round of the Pirate Game with 3 players without performing an intermediate measurement. }
\label{fig:pg_architecture3players_2nomeasure}
\end{figure}

The second stage in the game can be played in one of two ways. Either we perform an intermediate measuring step on the final state from the first round of voting as in Figure \ref{fig:pg_architecture3players_2measure}, or we ``ask'' what are the two player's strategy for this round without informing them of the results in the last round, like in the Figure \ref{fig:pg_architecture3players_2nomeasure}. In the original game the voting results are displayed in between rounds:

\begin{quotation}
If a majority or a tie is reached the goods will be allocated according to the proposal. Otherwise the proposer will be thrown overboard and the next pirate in the hierarchy assumes the place of the captain. 
\end{quotation}

When modelling this problem we want to study if withholding the results in the intermediate step will give way to new strategies.

\begin{figure}[h]
\centering 
\includegraphics[scale=0.55]{Figures/architecture/esquema/Slide5.png}
\caption{Overal view on the Quantum Pirate Game architecture. }
\label{fig:pg_architecture3players_architecture}
\end{figure}

\subsubsection{Utility}
\label{subsec:pirates_utility}

To build the expected payoff functionals for the three player situation we must take into account the sub-games created when the proposal fail. In Figure \ref{fig:pg_architecturegametree} we can see an extensive form representation of the game.

\begin{figure}[h]
\centering 
\includegraphics[scale=0.55]{Figures/architecture/GameTree/Slide1.png}
\caption{Extensive form representation of the game ($3$ player). Red circles represent failed proposals, green represent accepted proposals. }
\label{fig:pg_architecturegametree}
\end{figure}

As defined on Equation \ref{eq:quantum_game_definition_payoff_func},for each player we must specify a utility functional that attributed a real number to the measurement of the projection of a basis in the quantum state that we get after the game. This measurement can be understood as a probability of the system collapsing into that state (that derives from the Born Rule, Section \ref{subsubsec:bornrule}).


These utility functions will represent the degree of satisfaction for each pirate after game by atributiong a real number to a measurement performed to the system, as in Equation \ref{eq:quantum_game_definition_payoff_func} (Section \ref{sec:background_quantum_game_theory}). 

As each pirate wants to maximize her utility, the Nash equilibrium will be thoroughly used to find the strategies that the pirates will adopt\cite{nash50}\cite{Nash51}.

The real numbers used convey the logical relations of utility posed by the original problem description. Those numbers will represent the utility associated with the number of coins that a pirate gets, a death penalty, and a small incentive to climb the hierarchy.

The number of coins will translate directly the utility associated with getting those coins. For example if a pirate receives 5 gold coins and the proposal is accepted he will get a utility of 5. 

The highest ranking pirate in the hierarchy will be responsible to make a proposal to divide the 100 gold coins. This proposal is modelled as choosing some parameters for the payoff functionals for every player, according to some rules. For the initial step in the game with three pirates these parameters will be $\alpha_{1}, \alpha_{2}, \alpha_{3}$, and they will obey to the Equation \ref{eq:goodss}, where $k$ is the offest of the current captain, and $N$ the number of pirates in the game. 

\begin{equation}
\label{eq:goodss}
\sum_{i=k}^{N}\alpha_{i}=100, \forall i :\alpha_{i}\in\mathbb{N}_{0}
\end{equation}

The most interesting values for $(\alpha_{1}, \alpha_{2}, \alpha_{3})$ will be the allocation that results in a Nash equilibrium in the original Pirate Game $(99, 0, 1)$, and the case where the captain maximizes the number of coins he can get $(100, 0, 0)$. Will the game modelled as a quantum system allow the captain to acquire all the coins?

The proposed goods allocation will be executed if there is a majority (or a tie), in the voting step. A step in the game consists on the highest ranking pirate defining a proposal and the subsequent vote, where all players choose simultaneously an operator. 

If the proposal is rejected the captain will be thrown off board, to account for the fact that this situation is very undesirable for the captain he will receive a negative payoff of $-200$. This value was chosen to be much less than the highest number of coins a pirate could get.

\begin{quotation}
``When the result is indifferent the pirates prefer to throw another pirate overboard and thus climbing in the hierarchy.''
\end{quotation}

This means that the pirates have a small incentive to climb the hierarchy. For example in the three player classical game, the third player, who has the lowest rank, will prefer to defect the initial proposal if the player 1 doesn't give her a coin, even knowing that in the second round the player 2 will be able to keep the 100 coins. We will account for this preference by assigning an expected value of half a coin ($0.5$), to the payoff of the players that will climb on the hierarchy if the voting fails.

As stated in Section \ref{subsec:pirates_finalstate}, each step will be identified by the offset $k$, which will be equal to the offset $i$ for the highest ranking pirate. 

We can observe that in Equations \ref{eq:pirates_payoff32} and \ref{eq:pirates_payoff3} that we have two separate groups of outcomes: the one that represents outcomes where the proposal is passed and the other that aggregates the outcomes where the captain will be eliminated and the remaining players will keep playing.

So with $3$ pirates we start with step $1$ (Figure \ref{fig:pg_architecture3players_architecture}), if the voting is rejected we will get to step $2$, where 1 vote is enough to pass a proposal. The final payoff function (for example Equation \ref{eq:pirates_payoff32} for a $3$ player game), will be calculated recursively, the base case being the $2$ player sub-game in a $3$ player system will be Equation \ref{eq:pirates_payoff3}.

 

 \begin{equation}
 \begin{cases}
\begin{split}
E_{11}(\vert\psi_{fin1}\rangle, \alpha_{1})=\alpha_{1}\times(\vert\langle000\vert\psi_{fin1}\rangle\vert^{2} + \vert\langle100\vert\psi_{fin1}\rangle\vert^{2}
+ \vert\langle010\vert\psi_{fin1}\rangle\vert^{2}
+ \vert\langle001\vert\psi_{fin1}\rangle\vert^{2}
 ) - \\
 - 200\times(\vert\langle111\vert\psi_{fin1}\rangle\vert^{2} + \vert\langle110\vert\psi_{fin1}\rangle\vert^{2}
+ \vert\langle101\vert\psi_{fin1}\rangle\vert^{2}
+ \vert\langle011\vert\psi_{fin1}\rangle\vert^{2}
 )
\end{split}
\\
\begin{split}
E_{12}(\vert\psi_{fin1}\rangle, \alpha_{2})=\alpha_{2}\times(\vert\langle000\vert\psi_{fin1}\rangle\vert^{2} + \vert\langle100\vert\psi_{fin1}\rangle\vert^{2}
+ \vert\langle010\vert\psi_{fin1}\rangle\vert^{2}
+ \vert\langle001\vert\psi_{fin1}\rangle\vert^{2}
 ) - \\
 + (0.5 + E_{22})\times(\vert\langle111\vert\psi_{fin1}\rangle\vert^{2} + \vert\langle110\vert\psi_{fin1}\rangle\vert^{2}
+ \vert\langle101\vert\psi_{fin1}\rangle\vert^{2}
+ \vert\langle011\vert\psi_{fin1}\rangle\vert^{2}
 )
\end{split}
\\
\begin{split}
E_{13}(\vert\psi_{fin1}\rangle, \alpha_{3})=\alpha_{3}\times(\vert\langle000\vert\psi_{fin1}\rangle\vert^{2} + \vert\langle100\vert\psi_{fin1}\rangle\vert^{2}
+ \vert\langle010\vert\psi_{fin1}\rangle\vert^{2}
+ \vert\langle001\vert\psi_{fin1}\rangle\vert^{2}
 ) - \\
 + (0.5 + E_{23})\times(\vert\langle111\vert\psi_{fin1}\rangle\vert^{2} + \vert\langle110\vert\psi_{fin1}\rangle\vert^{2}
+ \vert\langle101\vert\psi_{fin1}\rangle\vert^{2}
+ \vert\langle011\vert\psi_{fin1}\rangle\vert^{2}
 )
\end{split}
\end{cases}
\label{eq:pirates_payoff32}
%\caption{Payoff funcionals for the $3$ player pirate game. The payoff function for the non-captain players is recursive because their decision to approve or reject the initial proposal will depend on how much they expect to gain in the next round ($E_{22}$ and $E_{23}$).}
\end{equation}

 \begin{equation}
\begin{cases}
\begin{split}
E_{22}(\vert\psi_{fin2}\rangle, \alpha_{2})=\alpha_{2}\times(\vert\langle000\vert\psi_{fin1}\rangle\vert^{2} + \vert\langle100\vert\psi_{fin1}\rangle\vert^{2}
+ \vert\langle010\vert\psi_{fin1}\rangle\vert^{2}
+ \vert\langle001\vert\psi_{fin1}\rangle\vert^{2}
 ) - \\
 + 0.5\times(\vert\langle111\vert\psi_{fin1}\rangle\vert^{2} + \vert\langle110\vert\psi_{fin1}\rangle\vert^{2}
+ \vert\langle101\vert\psi_{fin1}\rangle\vert^{2}
+ \vert\langle011\vert\psi_{fin1}\rangle\vert^{2}
 )
\end{split}
\\
\begin{split}
E_{23}(\vert\psi_{fin2}\rangle, \alpha_{3})=\alpha_{3}\times(\vert\langle000\vert\psi_{fin1}\rangle\vert^{2} + \vert\langle100\vert\psi_{fin1}\rangle\vert^{2}
+ \vert\langle010\vert\psi_{fin1}\rangle\vert^{2}
+ \vert\langle001\vert\psi_{fin1}\rangle\vert^{2}
 ) - \\
 + 100.5 \times(\vert\langle111\vert\psi_{fin1}\rangle\vert^{2} + \vert\langle110\vert\psi_{fin1}\rangle\vert^{2}
+ \vert\langle101\vert\psi_{fin1}\rangle\vert^{2}
+ \vert\langle011\vert\psi_{fin1}\rangle\vert^{2}
 )
\end{split}
\end{cases}
\label{eq:pirates_payoff3}
%\caption{Second round of a $3$ player game.The $2$ player subgame is the base case for this problem because one vote is enough to pass the proposal.}
\end{equation}


\subsection{Analysing the $2$-Player sub-game}
\label{subsec:2_player_subgame_piratas_doidos}

By hypothesis, the way we set up our system may originate different results, so when we have a $3$ player system, we will try to analyse the sub-game with $2$ players taking into account the original system. In the classical version of the Pirate Game, the problem with $2$ players is equivalent to a sub-problem with $2$ players of a problem with more players where previous captains are killed. 

\begin{table}
\begin{center}
\begin{tabular}{cc}
  a)\putindeepbox[7pt]{\includegraphics[scale=0.48]{compareluders/CC1.PNG}}
    & b)\putindeepbox[7pt]{\includegraphics[scale=0.48]{compareluders/CC2.PNG}} \\
  c)\putindeepbox[7pt]{\includegraphics[scale=0.48]{compareluders/CD1.PNG}}
    & d)\putindeepbox[7pt]{\includegraphics[scale=0.48]{compareluders/CD2.PNG}} \\
  e)\putindeepbox[7pt]{\includegraphics[scale=0.46]{compareluders/DC1.PNG}}
    & f)\putindeepbox[7pt]{\includegraphics[scale=0.48]{compareluders/DC2.PNG}} \\
  g)\putindeepbox[7pt]{\includegraphics[scale=0.46]{compareluders/DD1.PNG}}
    & h)\putindeepbox[7pt]{\includegraphics[scale=0.48]{compareluders/DD2.PNG}} \\
\end{tabular}
\caption{Comparison of the expected utilities when measuring and not measuring between steps 1 and 2 of a 3 player game (the player 1 will be the player 2 in the 3 player game, and the player 2 will be the player 3 in the 3 game situation).}
\label{tab:luderscomp}
\end{center}
 \end{table}



We set up a system with $3$ players and in the step $1$ the three players chose the operators $o_{10},o_{21},o_{31}$ producing the outcome $CDD$, the initial proposal for the gold coin division will be $(100, 0, 0)$. 

To analyse the step $2$, or the sub-game with $2$ pirates,  the system will be kept in the initial dimension ($\mathcal{H}^{3}$), because of the following reasons: if we have entangled states, they won't be are separable; There may be various mappings in a lower dimension that fit the higher dimension composition. This means that the previous highest ranking player will no longer have access to the voting operators, instead she will use invariably a symmetric coin operator $H$.

In one situation we will use the L\"{u}der's Rule, after the voting. This is the equivalent of performing a measurement on the system after the first round \ref{fig:pg_architecture3players_2measure}. In the other situation the two players will act on step $2$, as in Figure \ref{fig:pg_architecture3players_2nomeasure}. The results for this experiment (Table \ref{tab:luderscomp}), allow us to conclude that not measuring the system affects the expected utilities for the players in two of the four possible analysed outcomes. 

In this case the act of measuring leaves us with the same utility results one would get in a classic situation with $2$ players game, the equilibrium here is $CD$ outcome. 

Without the intermediate measuring step the outcome will depend on the $\gamma$ - the initial coefficient used to set up the system. For example:

\begin{itemize}
\item $\gamma = 0$, we have the same results as in the original problem. The Nash equilibrium is unique and it is $CD$.
\item $\gamma = \frac{\pi}{2}$, the equilibrium is $DC$, as the payoffs for the $CC$ outcome are $(-200, 0.5)$, and the outcome $DD$ will present $(100, 0)$ for the expected utilities of the players.
\item $\gamma = \frac{\pi}{4}$, all possible outcomes will be Nash equilibria. As we can see in Table \ref{tab:hate_myself}, no player can improve her payoff by unilaterally changing her strategy.



\end{itemize}

\begin{center}
\begin{table}
\begin{centering}
\begin{tabular}{ccc}
\hline 
 $\gamma = \frac{\pi}{4}$ & Player 2: C & Player 2: D\tabularnewline
\hline 
Player 1: C & (-50, 50.25) & (100, 0)\tabularnewline
Player 1: D & (100, 0) & (-50, 50.25)\tabularnewline
\hline 
\end{tabular}
\par\end{centering}

\caption{Normal form representation of the second step in a 3 player game with an entanglement coefficient of $\frac{\pi}{4}$, where there isn't an intermediate measuring. }
\label{tab:hate_myself}
\end{table}
\end{center}


\subsection{Analysis and Results}
\label{subsec:description_3}


If we measure in between steps where we measure the system in between states (as in Figure \ref{fig:pg_architecture3players_2measure}), we can replace the expected value for the subgame with $2$ players with the subgame perfect equilibrium in that case. The constants $(e_{22}, e_{23})$ will be the pair of utilities for player 2 and 3 in the subgame equilibrium will be $(e_{22}, e_{23})=(100, 0)$, in Equations \ref{eq:pirates_payoff2} and \ref{eq:pirates_payoff3}.  
 \begin{equation}
\begin{split}
E_{11}(\vert\psi_{fin1}\rangle, \alpha_{1})=\alpha_{1}\times(\vert\langle000\vert\psi_{fin1}\rangle\vert^{2} + \vert\langle100\vert\psi_{fin1}\rangle\vert^{2}
+ \vert\langle010\vert\psi_{fin1}\rangle\vert^{2}
+ \vert\langle001\vert\psi_{fin1}\rangle\vert^{2}
 ) - \\
 - 200\times(\vert\langle111\vert\psi_{fin1}\rangle\vert^{2} + \vert\langle110\vert\psi_{fin1}\rangle\vert^{2}
+ \vert\langle101\vert\psi_{fin1}\rangle\vert^{2}
+ \vert\langle011\vert\psi_{fin1}\rangle\vert^{2}
 )
\end{split}
\end{equation}

 \begin{equation}
\begin{split}
E_{12}(\vert\psi_{fin1}\rangle, \alpha_{2})=\alpha_{2}\times(\vert\langle000\vert\psi_{fin1}\rangle\vert^{2} + \vert\langle100\vert\psi_{fin1}\rangle\vert^{2}
+ \vert\langle010\vert\psi_{fin1}\rangle\vert^{2}
+ \vert\langle001\vert\psi_{fin1}\rangle\vert^{2}
 ) - \\
 + (0.5 + e_{22})\times(\vert\langle111\vert\psi_{fin1}\rangle\vert^{2} + \vert\langle110\vert\psi_{fin1}\rangle\vert^{2}
+ \vert\langle101\vert\psi_{fin1}\rangle\vert^{2}
+ \vert\langle011\vert\psi_{fin1}\rangle\vert^{2}
 )
\end{split}
\label{eq:pirates_payoff2}
\end{equation}

 \begin{equation}
\begin{split}
E_{13}(\vert\psi_{fin1}\rangle, \alpha_{3})=\alpha_{3}\times(\vert\langle000\vert\psi_{fin1}\rangle\vert^{2} + \vert\langle100\vert\psi_{fin1}\rangle\vert^{2}
+ \vert\langle010\vert\psi_{fin1}\rangle\vert^{2}
+ \vert\langle001\vert\psi_{fin1}\rangle\vert^{2}
 ) - \\
 + (0.5 + e_{23})\times(\vert\langle111\vert\psi_{fin1}\rangle\vert^{2} + \vert\langle110\vert\psi_{fin1}\rangle\vert^{2}
+ \vert\langle101\vert\psi_{fin1}\rangle\vert^{2}
+ \vert\langle011\vert\psi_{fin1}\rangle\vert^{2}
 )
\end{split}
\label{eq:pirates_payoff3}
\end{equation}

\begin{table}

\begin{center}
\begin{tabular}{cc}
  a)\putindeepbox[7pt]{\includegraphics[scale=0.46]{3Players/CDC100_0_0_1.PNG}}
    & a1)\putindeepbox[7pt]{\includegraphics[scale=0.46]{3Players/CDC100_0_0_2.PNG}} \\
\end{tabular}
\caption{Expected utility in a three player situation with intermediate measure step, where the players will use the $(Cooperate, Defect, Cooperate)$ operators. The initial proposal is $(\alpha_{1}, \alpha_{2}, \alpha_{3}) =(100, 0, 0)$.}
\label{tab:3player}
\end{center}
 \end{table}

In the Figure \ref{tab:3playerCCD99l}, the outcome $CDC$ with a proposal of $(\alpha_{1}, \alpha_{2}, \alpha_{3}) =(99, 0, 1)$ would represent the Nash Equilibrium of the classic Pirate Game (for $3$ players). 
%However depending on the angle $\gamma$ used to build the initial state .

\begin{table}
\begin{center}
\begin{tabular}{cc}
  a)\putindeepbox[7pt]{\includegraphics[scale=0.46]{3Players/CDC99_0_1_1.PNG}}
    & a1)\putindeepbox[7pt]{\includegraphics[scale=0.46]{3Players/CDC99_0_1_2.PNG}} \\
\end{tabular}
\caption{Expected utility for $3$ players with intermediate measure step, where the players will use the $(Cooperate, Defect, Cooperate)$ operators. The initial proposal is $(\alpha_{1}, \alpha_{2}, \alpha_{3}) =(99, 0, 1)$.}
\label{tab:3playerCCD99l}
\end{center}
 \end{table}











\begin{comment}

''''''Depending on the measurement outcome that occurs with probability x, the players will act on the second round. The L\"{u}der's rule is applied here. The highest ranking player will no longer have access to the voting operators, instead he will only be able to manipulate de system using a symmetric coin operator.

Keeping the system in a higher dimension is considered because of the following reasons: if we have entangled states, they won't be are separable, and there are various mappings to a lower dimension that fit the system composition. 


\begin{emph}
One important aspect is that the payoff functional changes

Initial state: variable to study.
Payoff functions: variable to study.
\end{emph}

\end{comment}