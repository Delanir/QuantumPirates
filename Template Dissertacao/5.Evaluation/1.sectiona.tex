\clearpage
\subsection{Discussion}
\label{subsec:3playergame:discution}

We tried to get our Quantum Model of the Pirate Game as close as possible in order to compare the the original game. 

In works such as the Prisoner's Dilemma\cite{Letters2002}\cite{Eisert2008}, and Quantum Ultimatum Game\cite{Fra2011}, the strategic space was analysed allowing a infinity of mixed quantum strategies.

When designing the Quantum Model, we claimed that allowing mixed quantum strategies would degenerate the voting problem, as the original problem has stages that are dependent on the actions taken by the players.

The set-up of the initial system was crucial to introduce the phenomenon of entanglement which was the principal study variable in our model.

In the simulation of the Quantum Pirate Game with $3$ players we found possible final states where the probability distributions of the final outcomes depended on the entanglement parameter $\gamma$, but the expected utility functions did not vary with $\gamma$. This particular result is an example that arises from the problem description (the way the expected utility functionals are build for each player), however it is not explored in the related work. 

It is impossible to not trace a parallel with our world that seems to have some deterministic rules from interactions governed by quantum mechanics at a fundamental level. The concept of temperature is an example of that. At a fundamental level the temperature is the measurement of radiation emitted by oscillating particles. 


This version of the Quantum Pirate Game ended up being a case study of how to modify as little as possible a classical Game Theory problem while introducing mechanics inspired in the Quantum Theory. 

 






