% %%%%%%%%%%%%%%%%%%%%%%%%%%%%%%%%%%%%%%%%%%%%%%%%%%%%%%%%%%%%%%%%%%%%%%
% Conclusions and Future Work:
% %%%%%%%%%%%%%%%%%%%%%%%%%%%%%%%%%%%%%%%%%%%%%%%%%%%%%%%%%%%%%%%%%%%%%%
\fancychapter{Conclusions and Future Work}
\label{cap:conclusions}

\textit{This section closes this document. It provides an overview on the work, presents a summary of the main results and relevant contributions. Moreover we compiled a list of interesting points worth pursuing in the future.}


\section{Future Work}
\label{sec:5FutureWork}

The principles of quantum computing provide an extremely rich source of ideas to extend other fields of knowledge.
Some suggestions for possible extensions for this work would be:

\begin{enumerate}
\item \textbf{To implement and test this quantum model with human subjects.} The field of Quantum Game theory tries to explain the Human reasoning process by using principles from quantum game theory, namely quantum probabilities. To expose subjects to a quantum model and having them try to take advantage of the rules of the system might provide valuable insights to the way we reason. 
\item \textbf{A graphical user interface (GUI) of the model.} This would provide a more approachable way to tweak parameters.
\item \textbf{To build a framework that .} Although there are already some frameworks that facilitate the creation of a quantum models of a game theory problem, there is still not a comprehensive .
\item \textbf{To create and structure a platform to promote scientific divulgation of quantum computing.} While investigating and developing this solution, there was often the thought that it would be important for undergraduate Computer Science students to come with contact with this paradigm. This field is still shrouded with mystery to many people, and this makes it a priority topic for scientific divulgation. In this work we tried to present examples that someone with a basic understanding of Algebra may be able to follow, but recollecting my experiences this is by no means enough.
\end{enumerate}

\cleardoublepage