% %%%%%%%%%%%%%%%%%%%%%%%%%%%%%%%%%%%%%%%%%%%%%%%%%%%%%%%%%%%%%%%%%%%%%%
% Conclusions and Future Work:
% %%%%%%%%%%%%%%%%%%%%%%%%%%%%%%%%%%%%%%%%%%%%%%%%%%%%%%%%%%%%%%%%%%%%%%
\fancychapter{Conclusions and Future Work}
\label{cap:conclusions}

\textit{This section closes this document. It provides an overview on the work, presents a summary of the main results and relevant contributions. Moreover we compiled a list of interesting points worth pursuing in the future.}

From an historic point of view there is a character who is active on the fields of Quantum Mechanics, Game Theory, and Computer Science (among others), John von Neumman. von Neumman died at the age of $53$ years, so we can only specculate if he would ever try to apply the principles of Quantum Mechanics to Game Theory. 


Robel Laughlin uses the concept of emergence to explain the way classical phenomena arises from quantum mechanics\cite{Laughlin2005}. In the Pirate Game we found that derived from the game design and the strategic space (the players could only use pure strategies like Cooperate/Defect), there were possible outcomes to the game where the entanglement parameter did not affect the final result.

\section{Future Work}
\label{sec:5FutureWork}

The principles of quantum computing provide an extremely rich source of ideas to extend other fields of knowledge.
Some suggestions for possible extensions for this work would be:

\begin{enumerate}

\item \textbf{Optimize this Quantum Pirate Game Model.} To model this game as a Quantum Bayesian Network. The Bayesian Networks (also referred to as Belief Networks, Probabilistic networks or causal networks) consist of an Directed Accyclic Graph to represent a set of conditional dependencies between random variables, that provide a compact description that allows to calculate a joint probability distribution, having in mind those dependencies into account. \cite{Tucci2012} proposes a model for Quantum Bayesian Networks, so designing a Quantum Bayesian Network approximation for the Quantum Pirate Game, having the description of this model as a starting point, could be a relevant research topic.

\item \textbf{To implement and test this quantum model with human subjects.} The field of Quantum Cognition tries to explain the Human reasoning process by using principles from quantum game theory, namely quantum probabilities. To expose subjects to a quantum model and having them try to take advantage of the rules of the system might provide valuable insights to the way we reason. 

\item \textbf{Increase the number of players.} Studying the quantum system with $8$ players and $1$ gold coin, would be an interesting extentions for this work. This would allow to experiment the bizarre survival sittuation that happens when there are not enough coins for the captain to bribe the other pirates. However this particular case of the pirate game would need a $35$-qubit system in order to be studied ($8$ qubits for the first stage, $7$ for the second stage, until reaching a $2$-player sub-game). In a classical computer and full joint probability this would mean working with vectors in the scale $10^{10}$.

\item \textbf{A graphical user interface (GUI) for the Pirate Game.} This would provide a more approachable way to tweak parameters.

\item \textbf{To create and structure a platform to promote scientific divulgation of quantum computing.} While investigating and developing this solution, there was often the thought that it would be important for undergraduate Computer Science students to come with contact with this paradigm. This field is still shrouded with mystery to many people, and this makes it a priority topic for scientific divulgation. In this work we tried to present examples that someone with a basic understanding of Algebra may be able to follow.
\end{enumerate}

\cleardoublepage