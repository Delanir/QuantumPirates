% %%%%%%%%%%%%%%%%%%%%%%%%%%%%%%%%%%%%%%%%%%%%%%%%%%%%%%%%%%%%%%%%%%%%%%
% Conclusions and Future Work:
% %%%%%%%%%%%%%%%%%%%%%%%%%%%%%%%%%%%%%%%%%%%%%%%%%%%%%%%%%%%%%%%%%%%%%%
\fancychapter{Conclusions and Future Work}
\label{cap:conclusions}

\textit{This section closes this document. It provides an overview on the work, presents a summary of the main results and relevant contributions. Moreover we compiled a list of interesting points worth pursuing in the future.}

From an historic point of view there is a character who is active on the fields of Quantum Mechanics, Game Theory, and Computer Science (among others), John von Neumman. von Neumman died at the age of $53$ years. 

\section{Future Work}
\label{sec:5FutureWork}

The principles of quantum computing provide an extremely rich source of ideas to extend other fields of knowledge.
Some suggestions for possible extensions for this work would be:

\begin{enumerate}
\item \textbf{Study different ways to set up the initial state.} Setting the initial state in a Quantum Game Theory game is not trivial. What if the pirates had a prior pre-disposition to cooperate or defect? Possibly exploring various initial superpositions derived from the players pre-disposition to cooperate or defect would provide insights on decision making with non-rational players. 
\item \textbf{To implement and test this quantum model with human subjects.} The field of Quantum Game theory tries to explain the Human reasoning process by using principles from quantum game theory, namely quantum probabilities. To expose subjects to a quantum model and having them try to take advantage of the rules of the system might provide valuable insights to the way we reason. 
\item \textbf{Increase the number of players.} Studying the quantum system with $8$ players and $1$ gold coin, would be an interesting extentions for this work. This would allow to experiment the bizarre survival sittuation that happens when there are not enough coins for the captain to bribe the other pirates.
\item \textbf{A graphical user interface (GUI) for the Pirate Game.} This would provide a more approachable way to tweak parameters.
\item \textbf{To create and structure a platform to promote scientific divulgation of quantum computing.} While investigating and developing this solution, there was often the thought that it would be important for undergraduate Computer Science students to come with contact with this paradigm. This field is still shrouded with mystery to many people, and this makes it a priority topic for scientific divulgation. In this work we tried to present examples that someone with a basic understanding of Algebra may be able to follow, but recollecting my experiences it is not enough.
\end{enumerate}

\cleardoublepage