% %%%%%%%%%%%%%%%%%%%%%%%%%%%%%%%%%%%%%%%%%%%%%%%%%%%%%%%%%%%%%%%%%%%%%%
% Conclusions and Future Work:
% %%%%%%%%%%%%%%%%%%%%%%%%%%%%%%%%%%%%%%%%%%%%%%%%%%%%%%%%%%%%%%%%%%%%%%
\fancychapter{Conclusions and Future Work}
\label{cap:conclusions}

\textit{This section closes this document. It provides an overview on the work, presents a summary of the main results and relevant contributions. Moreover we compiled a list of interesting points worth pursuing in the future.}

From an historic point of view there is a character who is active on the fields of Quantum Mechanics, Game Theory, and Computer Science (among others), John von Neumman. von Neumman died at the age of $53$ years, so we can only speculate if he would ever try to apply the principles of Quantum Mechanics to Game Theory. 

%This dissertation aimed at providing a multi-disciplinary


% c 

With this dissertation we developed an approach to a novel Quantum Game - the Pirate Game. While defining the quantum model for the game we found that the most unique features of the problem, that appear when the number of coins does not allow for the captain to bribe the necessary votes in order for him to be spared, would pose an hard problem to analyse because of the exponential growth the system experiences when we increase the number of qubits. Possibly with the advent of the first commercial quantum computers these we will have resources to simulate these kinds of systems in a more spacial optimized way - the physical model for a qubit can be the polarization of a photon. 

Nevertheless we developed a model for the Quantum Pirate Game with $3$ players, and we also studied the system for $2$ players. Both games are strictly determined in the original problem, however when we allow the system to be entangled we start noticing that  there are quantum operators that can cause interference. When the game is maximally entangled, we could not find a pure strategy Nash Equilibrium. These results are similar to the findings in models such as the Prisoner's Dilemma\cite{Eisert2008}\cite{Letters2002}, and Stackelberg Duopoly \cite{Khan2011}.

We also found that, when the system is maximally entangled, if we restrict the strategic space for some players we break the cycle of optimal quantum counter-strategies suggested by \cite{Du}: when the entanglement is maximal, $\mathcal{U}(\pi, 0)$ is the optimal counter-strategy for $C$ (represented by the identity matrix), $\mathcal{U}(0, \frac{\pi}{2})$ is the optimal counter strategy for $\mathcal{U}(\pi, 0)$, $D$ becomes the optimal counter-strategy for $\mathcal{U}(0, \frac{\pi}{2})$, and $C$ becomes an optimal counter strategy for $D$. This means that it is possible for a player to explore this weakness if other players are unaware. If we had a quantum gambling system, or a voting system such as in the Pirate Game, the strategic space and the entanglement are information that needs to be available for all players equally in order for them to choose their best-response.

Another contribution from this thesis is the simulation code in Matlab provided in the appendices, these examples might be used in order to experiment the quantum models of the Prisoner's Dilemma, a Quantum Roulette, and finally the Quantum Pirate Game. We also developed a Matlab simulation of the Quantum algorithms Walk in a Line. These simulations were built while analysing related work on the area.

The implications of Quantum Mechanics are still very detached from the way we think. The Quantum theory is deeply rooted in the concept of probability. Robert Laughlin uses the concept of emergence to explain the way classical phenomena arises from quantum mechanics\cite{Laughlin2005}. It is almost poetic that in our apparently deterministic world arises with all its patterns surges from the chaos. 

\section{Future Work}
\label{sec:5FutureWork}

The principles of quantum computing provide an extremely rich source of ideas to extend other fields of knowledge.
Some suggestions for possible extensions for this work would be:

\begin{enumerate}

\item \textbf{To make conditional measurements by applying the L\"{u}der's Rule} We mentioned a proposal for an extensive form game that uses measurements to separate sequential stages in the game in Section \ref{subsec:ultimatum}, while discussing a quantum approach to the Ultimate Game. The objective would be to separate each stage of the game with a measurement operation. After a stage where the players made their strategic moves, the resulting quantum state would be de-entangled, projected in order to separate  the sub-game where the actual proposal was accepted and the complementary state where the proposal was not accepted. The game would be then re-entangled with a new entanglement coefficient. 

\item \textbf{Optimize this Quantum Pirate Game Model.} To model this game as a Quantum Bayesian Network. The Bayesian Networks (also referred to as Belief Networks, Probabilistic networks or causal networks) consist of an Directed Acyclic Graph to represent a set of conditional dependencies between random variables, that provide a compact description that allows to calculate a joint probability distribution, having in mind those dependencies into account. \cite{Tucci2012} proposes a model for Quantum Bayesian Networks, so designing a Quantum Bayesian Network approximation for the Quantum Pirate Game, having the description of this model as a starting point, could be a relevant research topic.

\item \textbf{To implement and test this quantum model with human subjects.} The field of Quantum Cognition tries to explain the Human reasoning process by using principles from quantum game theory, namely quantum probabilities. To expose subjects to a quantum model and having them try to take advantage of the rules of the system might provide valuable insights to the way we reason. 

\item \textbf{Increase the number of players.} Studying the quantum system with $8$ players and $1$ gold coin, would be an interesting extensions for this work. This would allow to experiment the bizarre survival situations that happens when there are not enough coins for the captain to bribe the other pirates. However this particular case of the pirate game would need a $35$-qubit system in order to be studied ($8$ qubits for the first stage, $7$ for the second stage, until reaching a $2$-player sub-game). In a classical computer and full joint probability this would mean working with vectors in the scale $10^{10}$.

\item \textbf{A graphical user interface (GUI) for the Pirate Game.} This would provide a more approachable way to tweak parameters.

\item \textbf{To create and structure a platform to promote scientific dissemination of quantum computing.} While investigating and developing this solution, there was often the thought that it would be important for undergraduate Computer Science students to come with contact with this paradigm. This field is still shrouded with mystery to many people, and this makes it a priority topic for scientific promulgation. In this work we tried to present examples that someone with a basic understanding of Algebra may be able to follow.
\end{enumerate}

\cleardoublepage