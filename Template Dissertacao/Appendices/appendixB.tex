\fancychapter{Quantum Prisioner's Dillema}
\label{ap:b}


\begin{lstlisting}
% Quantum Prisioner's Dillema
function out = quantumprisionersdillema(g,T,R,P,S)
%quantumprisionersdillema(g)
%   
%                Simulates the payoff of 2-player prisioner's dillema game
%                 
%        IN:
%                g   : entanglement coefitient, by defaut g=0 (no
%                entanglement)
%                T   : temptation, by defaut T=3
%                R   : reward, by defaut R=2
%                P   : punishment, by defaut P=1
%                S   : suckers, by defaut S=0
%
% To have a prisoner's dilemma game in the cannonical form, 
% it must respect:
% T > R > P > S
%            P      Player 2
%            l       C       D
%            a  C: (R, R)  (S, T)
%            y  D: (T, S)  (P, P)
%            e
%            r
%            1
%
%        OUT:    
%                out: 4x2 matrix, column 1 has the expected utility for
%                     player 1, column 2 has the expected utility for
%                     player 2. A possible outcome corresponds to each
%                     line like [CC;CD;DC;DD]
%                     
% All parameters are optional
%------------------------%

%-- Entanglement coeficient (checks if exists)
%-- Check variables and set to defaults
if exist('g','var')~=1, g=0; end

%-- Utility
%-- Check variables and set to defaults
if exist('T','var')~=1, T=3; end
if exist('R','var')~=1, R=2; end
if exist('P','var')~=1, P=1; end
if exist('S','var')~=1, S=0; end

%-- Actions
%   cooperate= [1 0;0 1]
C= eye(2);
%   defect= [0 1;1 0]
D= ones(2)-eye(2);

%-- Building the initial state
ini= cos(g)*kron([1 0]',[1 0]') + 1i*sin(g)*kron([0 1]',[0 1]');
out= zeros(4,2);   
%-- Simulation a outcome where player1 (C)ooperates 
%   and player 2 (C)ooperates
   finCC= kron(C,C)*ini;
   out(1,1)=payofffunc_player1(finCC, T, R, P, S);
   out(1,2)=payofffunc_player2(finCC, T, R, P, S);
%-- Simulation a outcome where player1 (C)ooperates 
%   and player 2 (D)efects
   finCD= kron(C,D)*ini;
   out(2,1)=payofffunc_player1(finCD, T, R, P, S);
   out(2,2)=payofffunc_player2(finCD, T, R, P, S);
%-- Simulation a outcome where player1 (D)efects
%   and player 2 (C)ooperates
   finDC= kron(D,C)*ini;
   out(3,1)=payofffunc_player1(finDC, T, R, P, S);
   out(3,2)=payofffunc_player2(finDC, T, R, P, S);
%-- Simulation a outcome where player1 (D)efects
%   and player 2 (D)efects
   finDD= kron(D,D)*ini;
   out(4,1)=payofffunc_player1(finDD, T, R, P, S);
   out(4,2)=payofffunc_player2(finDD, T, R, P, S);
end

function u = payofffunc_player1(fin, T, R, P, S)
%payofffunc_player1(fin)
%   
%                Calculates the payoff for player 1
%              IN
%                   fin: final state
%                   T, R, P, S: payoff numbers
%              OUT
%                   u: expected utility for player 1
    u= R*abs(dot([1 0 0 0]',fin))^2 + T*abs(dot([0 0 1 0]',fin))^2+ P*abs(dot([0 0 0 1]',fin))^2;
end

function u = payofffunc_player2(fin, T, R, P, S)
%payofffunc_player1(fin)
%   
%                Calculates the payoff for player 2
%              IN
%                   fin: final state
%                   T, R, P, S: payoff numbers
%              OUT
%                   u: expected utility for player 2
    u= R*abs(dot([1 0 0 0]',fin))^2 + T*abs(dot([0 1 0 0]',fin))^2+ P*abs(dot([0 0 0 1]',fin))^2;
end

\end{lstlisting}