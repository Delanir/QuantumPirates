\fancychapter{Results: Pirate Game}
\label{ap:d}

\section{$2$ Player Game}

\subsection{Simulation}
\begin{lstlisting}
% Quantum Pirate Game
function out = quantum2piratesanalise()
    C= eye(2);
%   defect= [0 1;1 0]
D= ones(2)-eye(2);

Dy= D;
Dy(2,1)=-1;
Dz= [1i 0;0 -1i];

g=pi/2;

u1=zeros(4,4);
u2=zeros(4,4);

for s1=1:1:4
    
    
    switch s1
    case 1
        U1=C;
    case 2
        U1=D;
    case 3
        U1=Dy;
    otherwise
        U1=Dz;
end
for s2=1:1:4
    
   switch s1
    case 1
        U2=C;
    case 2
        U2=D;
    case 3
        U2=Dy;
    otherwise
        U2=Dz;
           
   end

   out=quantum2pirates(g,U1,U2);
    u1(s1,s2)=out(1);
    u2(s1,s2)=out(2);
   
end

end

figure


h=bar3(u1);

% set(h(1),'facecolor','green');
% set(h(2),'facecolor','blue');
% set(h(3),'facecolor','red');
%mesh(prob);
%

set(gca,'yTickLabel',{'C', 'D',  'U(\pi ,0)', 'U(0, \pi /2)'})
% set(gca,'xTick',0:length(t)/4:length(t))
set(gca,'xTickLabel',{'C', 'D',  'U(\pi ,0)', 'U(0, \pi /2)'})
ylabel('U_2');
xlabel('U_1');
zlabel('E_1');

title('\gamma = \pi /2 Expected utility for player 1 (E_1)')

figure


h=bar3(u2);

% set(h(1),'facecolor','green');
% set(h(2),'facecolor','blue');
% set(h(3),'facecolor','red');
%mesh(prob);
%

set(gca,'yTickLabel',{'C', 'D',  'U(\pi ,0)', 'U(0, \pi /2)'})
% set(gca,'xTick',0:length(t)/4:length(t))
set(gca,'xTickLabel',{'C', 'D',  'U(\pi ,0)', 'U(0, \pi /2)'})
ylabel('U_2');
xlabel('U_1');
zlabel('E_2');

title('\gamma = \pi /2 Expected utility for player 2 (E_2)')

end

function out = quantum2pirates(g,U1,U2)
%quantumprisionersdillema(g)
%   
%                Simulates the payoff of 2-player prisioner's dillema game
%                 
%        IN:
%                g   : entanglement coefitient, by defaut g=0 (no
%                entanglement)
%               
%                U1   : Player 1 strategy
%                U2   : Player 2 strategy
%
% All parameters are optional
%------------------------%

%-- entanglement coeficient (checks if exists)
%-- Check variables and set to defaults
if exist('g','var')~=1, g=0; end

%-- Utility
%-- Check variables and set to defaults
a22=100;
a23=0;
death=-200;
incentive=0.5;


%-- Strategies
if exist('U1','var')~=1, U1=[0 1i;1i 0]; end
if exist('U2','var')~=1, U2=[0 1i;1i 0]; end

%-- Actions
%   cooperate= [1 0;0 1]
C= eye(2);
%   defect= [0 1;1 0]
D= ones(2)-eye(2);

%-- Building the initial state
ini= cos(g/2)*kron([1 0]',[1 0]') + 1i*sin(g/2)*kron([0 1]',[0 1]');
%-- Deentangles to produce a final state
Jt = ctranspose( expm(1i*(g/2)*kron(D,D)));

   fin= Jt*kron(U1,U2)*ini;
   u1=payofffunc_player1(fin, a22, a23, death, incentive);
   u2=payofffunc_player2(fin, a22, a23, death, incentive);
    out= [u1 u2];
end

function u = payofffunc_player1(fin, a22, a23, death, incentive)
%payofffunc_player1(fin)
%   
%                Calculates the payoff for player 1
%              IN
%                   fin: final state
%                   a22, a23, death, incentive: payoff coefficients
%              OUT
%                   u: expected utility for player 1
%(norm(conj([0 1 1 1 0 1 1 1]').*fin2CC))^2
    u= a22*(measure2([1 0 0 0]',fin)+measure2([0 1 0 0]',fin)+measure2([0 0 1 0]',fin) )-200*measure2([0 0 0 1]',fin) ;
end

function u = payofffunc_player2(fin, a22, a23, death, incentive)
%payofffunc_player1(fin)
%   
%                Calculates the payoff for player 2
%              IN
%                   fin: final state
%                   a22, a23, death, incentive: payoff coefficients
%              OUT
%                   u: expected utility for player 2
    u= a23*(measure2([1 0 0 0]',fin)+measure2([0 1 0 0]',fin)+measure2([0 0 1 0]',fin) )+(a22+a23+incentive)*measure2([0 0 0 1]',fin) ;
end

function m = measure2(b,fin)
m=abs( sum(b.*conj(fin)))^2;
end



\end{lstlisting}

\section{$3$ Player Game}

\subsection{Simulation}
\begin{lstlisting}


function out = simulationQuantumPirateGame3Players(a_11,a_12,a_13,a_22,a_23, gamma, U1,U2,U3,U4,U5)
%
%        IN:
%-- alocation proposal a_ij -i number of the player that proposes a_ij coins
%   to player j
%-- gamma entanglemen coefitient
%-- U1, U2, U3, U4, U5 player's strategies; default CDCCD
%
%       OUT:
%-- expected utility for players 1 ,2 ,3:  [u1 u2 u3]

%-- Check variables and set to defaults
if exist('a_11','var')~=1, a_11=99; end
if exist('a_12','var')~=1, a_12=0; end
if exist('a_13','var')~=1, a_13=1; end
if exist('a_22','var')~=1, a_22=100; end
if exist('a_23','var')~=1, a_23=0; end
if exist('gamma','var')~=1, gamma=0; end

%-- Classic Pure Strategy Operators
%   cooperate= [1 0;0 1]
C= eye(2);
%   defect= [0 1;1 0]
D= ones(2)-eye(2);
D(2,1)=1;

if exist('U1','var')~=1, U1=C; end
if exist('U2','var')~=1, U2=D; end
if exist('U3','var')~=1, U3=C; end
if exist('U4','var')~=1, U4=C; end
if exist('U5','var')~=1, U5=D; end

%-- Matrix containing all pure-states for hilbert space 32
H32B=eye(32);

step=0.1;
t=0:step:pi;

i=gamma;

%-- varying the entanglement parameter

%-- Building the initial state, for the entanglement parameter i
ini= cos(i/2)*kron([1 0]', kron([1 0]',[1 0]'))+1i*sin(i/2)*kron([0 1]', kron([0 1]',[0 1]'));
%-- Entanglement Gate J
J= expm(1i*(i/2)*kron(D,kron(D,kron(D,kron(D,D)))));
%-- Alternative way to build the initial state
ini= J*H32B(:,1);
%-- Deentangles to produce a final state
Jd = ctranspose( J);

H= fft(eye(2))/sqrt(2);
fin = kron(U1,kron(U2,kron(U3,kron(U4,U5))))*ini;
fin= Jd*fin;
out = expectedUtility(fin, a_11,a_12,a_13,a_22,a_23);

end


function out = expectedUtility(fin, a_11,a_12,a_13,a_22,a_23)
%payofffunc_player1(fin)
%
%                Calculates the payoff for player 1
%              IN
%                   fin: final state
%

%-- Check variables and set to defaults
if exist('a_11','var')~=1, a_11=99; end
if exist('a_12','var')~=1, a_12=0; end
if exist('a_13','var')~=1, a_13=1; end
if exist('a_22','var')~=1, a_22=100; end
if exist('a_23','var')~=1, a_23=0; end

%-- Matrix containing all pure-states for hilbert space 32
H32B=eye(32);

prob_proposal_1_accepted=0;
for accepted=[1 2 3 4 5 6 7 8 9 10 11 12 17 18 19 20]
    prob_proposal_1_accepted= prob_proposal_1_accepted + measure(H32B(:,accepted),fin);
end
prob_proposal_1_accepted;

prob_proposal_1_rejected=0;
for rejected=[13 14 15 16 21 22 23 24 25 26 27 28 29 30 31 32]
    prob_proposal_1_rejected= prob_proposal_1_rejected + measure(H32B(:,rejected),fin);
end
prob_proposal_1_rejected;

prob_proposal_2_accepted=0;
for accepted=[13 14 15 21 22 23 25 26 27 29 30 31]
    prob_proposal_1_accepted= prob_proposal_1_accepted + measure(H32B(:,accepted),fin);
end
prob_proposal_2_accepted;

prob_proposal_2_rejected=0;
for rejected=[16 24 28 32]
    prob_proposal_1_rejected= prob_proposal_1_rejected + measure(H32B(:,rejected),fin);
end
prob_proposal_2_rejected;

u1 = a_11*prob_proposal_1_accepted -200* prob_proposal_1_rejected;

u2 = a_12*prob_proposal_1_accepted +(0.5 + a_22)*prob_proposal_2_accepted -199.5*prob_proposal_2_rejected;

u3 = a_13*prob_proposal_1_accepted +(0.5 + a_23)*prob_proposal_2_accepted + 100.5*prob_proposal_2_rejected;
out =[ u1 u2 u3];
end

function m = measure(b,fin)
m= (norm(conj(b).*fin))^2;
end


function u= U_thetha_phi(theta,phi)
u= [exp(1i*phi)cos(theta/2) *sin(theta/2);sin(theta/2) -exp(-i*phi)*cos(theta/2)];
end
\end{lstlisting}

